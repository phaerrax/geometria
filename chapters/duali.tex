\chapter{Spazi duali}
Tra le applicazioni lineari su uno spazio vettoriale, sono particolarmente interessanti le \emph{forme} lineari, o \emph{funzionali} lineari, ossia le applicazioni da $V$ al campo $K$ su cui è definito.
Lo spazio (anch'esso vettoriale) dei funzionali lineari è detto \emph{spazio duale} di $V$, e si indica tradizionalmente con $V^*$.
Per quanto visto nei capitoli precedenti $V^*$ è isomorfo a $\mat(1,\dim V,K)$, in quanto questi funzionali sono associati, scelta una base di $V$, a delle matrici con una sola riga: vediamo subito allora che $\dim V^*=\dim V$.

Ecco alcuni esempi importanti di funzionali lineari.
\begin{itemize}
	\item In $\R^n$, possiamo definire un iperpiano come il luogo degli zeri di una funzione
		\begin{equation*}
			f(x_1,\dots,x_n)=a_1x_1+\cdots+a_nx_n-b
		\end{equation*}
		per dei coefficienti $a_i\in\R$ non tutti nulli e $b\in\R$.
		In generale, per uno spazio vettoriale $V$ qualunque, non disponiamo di una base canonica, quindi non possiamo trovare immediatamente i coefficienti $x_1,\dots,x_n$ dell'equazione $f(x_1,\dots,x_n)=0$.
		I funzionali lineari forniscono un modo per caratterizzare gli iperpiani senza bisogno di coordinate (cioè di una base), ossia in modo \emph{canonico}.
		Possiamo interpretare l'equazione come l'applicazione di un funzionale $\alpha$ del duale di $\R^n$ sul vettore $v=[x_1\,\cdots\, x_n]^T$, ossia come $\alpha(v)=b$.
		Per un generico spazio $V$ possiamo definire un iperpiano come l'insieme $\{v\in V\colon \alpha(v)=b\}$ per un dato funzionale $\alpha\in V^*$.
	\item Nello spazio delle funzioni continue reali in $[0,1]$, una mappa da esso a $\R$ può essere ad esempio 
		\begin{equation*}
			f\mapsto\int_0^1f(t)\,\dd t
		\end{equation*}
		che è evidentemente lineare.
		Possiamo definire un funzionale lineare anche a partire dalle funzioni stesse di $\cont{}\big([0,1]\big)$: ad una funzione $g$ di tale spazio associamo il funzionale
		\begin{equation*}
			T_g\colon f\mapsto\int_0^1f(t)g(t)\,\dd t.
		\end{equation*}
		Anche il funzionale definito precedentemente è di questo tipo, con $g=1$.
\end{itemize}

\section{Base duale}
\label{sec:base-duale}
Vediamo ora quelli che probabilmente sono i funzionali lineari più importanti di $V^*$.
Scegliamo una base $\mathcal B=\{b_1,\dots,b_n\}$ per lo spazio vettoriale $V$ sul campo $K$.
Sappiamo che un'applicazione lineare è univocamente determinata dalla sua azione su una base, dunque lo stesso vale per i funzionali lineari.
Definiamo $\beta_i\colon V\to K$ come
\begin{equation}
	\beta_i(b_j)=\delta_{ij}
	\label{eq:funzionale-coordinate}
\end{equation}
per $i,j=1,\dots,n$.
In altre parole, il funzionale $\beta_i$ ``estrae'' la $i$-esima coordinata di un vettore: se $v=\sum_{j=1}^nv_jb_j$ allora $\beta_i(v)=v_i$.

Consideriamo ora l'insieme $\mathcal B^*=\{\beta_1,\dots,\beta_n\}$: preso un qualsiasi $\alpha\in V^*$, possiamo scriverlo come combinazione lineare dei $\beta_i$ osservando che se $\alpha=\sum_{k=1}^n\alpha_k\beta_k$ allora
\begin{equation}
	\alpha(b_j)=\sum_{k=1}^n\alpha_k\beta_k(b_j)=\sum_{k=1}^n\alpha_k\delta_{kj}=\alpha_j
\end{equation}
e ciò identifica i coefficienti di $\alpha$ rispetto a $\mathcal B^*$.
Preso allora $v=\sum_{i=1}^nv_ib_i$, si ha
\begin{equation}
	\alpha(v)=\sum_{i=1}^nv_i\alpha(b_i)=\sum_{i=1}^nv_i\alpha_i
\end{equation}
e allo stesso tempo
\begin{equation}
	\sum_{i=1}^n\alpha_i\beta_i(v)=\sum_{i=1}^n\alpha_i\beta_i\bigg(\sum_{j=1}^nv_jb_j\bigg)=\sum_{i=1}^n\alpha_i\sum_{j=1}^nv_j\delta_{ij}=\sum_{i=1}^n\alpha_iv_i
\end{equation}
che coincide con la precedente.
Questo verifica che qualsiasi funzionale lineare può essere espresso come combinazione lineare dei $\beta_i$.
Inoltre, se $\sum_{i=1}^n\mu_i\beta_i=0$, cioè è il funzionale nullo, allora sulla base $\mathcal B$ risulta
\begin{equation}
	0=\sum_{i=1}^n\mu_i\beta_i(b_j)=\sum_{i=1}^n\mu_i\delta_{ij}=\mu_j
\end{equation}
per $j=1,\dots,n$ quindi $\mathcal B^*$ è anche linearmente indipendente: ciò prova che è una base di $V^*$.\footnote{
	Sapendo già che la dimensione di $V^*$ è $n$, l'indipendenza lineare di $\mathcal B^*$ sarebbe stata sufficiente a mostrare che è una base di $V^*$, ciononostante questa dimostrazione è utile per capire come individuare i coefficienti di un funzionale rispetto alla base $\mathcal B^*$.
}
\begin{definizione} \label{d:base-duale}
	Data una base $\mathcal B=\{b_1,\dots,b_n\}$ di uno spazio vettoriale $V$, la base $\{\beta_1,\dots,\beta_n\}$ di $V^*$ tale che per ogni $i,j=1,\dots,n$
	\begin{equation}
		\beta_i(b_j)=\delta_{ij}
	\end{equation}
	è detta \emph{base duale} di $\mathcal B$.
\end{definizione}
Come già visto, in questa base la $i$-esima componente di un funzionale lineare $\alpha$ è data da $\alpha(b_i)$.
Va detto che non ha senso parlare di una ``base duale di $V^*$'', dato che non è unica: ad ogni scelta di una base di $V$ corrisponde una base duale per $V^*$, a essa associata.

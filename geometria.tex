\documentclass[a4paper,notitlepage]{book}

\usepackage{mathematics}
\hypersetup{
	pdftitle={Geometria 1},
	pdfauthor={Ferracin Davide},
	pdflang={it},
	pdfsubject={Appunti del corso di Geometria 1},
	pdfcreator={Davide Ferracin},
	pdfcaptionwriter={Davide Ferracin}
	pdfcopyright={Creative Commons Attribution-ShareAlike 4.0 International},
	pdflicenseurl={http://creativecommons.org/licenses/by-sa/4.0/},
	pdfmetalang={it}
}

%titolo e pagina iniziale
\title{
	{\sffamily\fontsize{35}{42}\selectfont Geometria 1}
}
\author{
	{\small a cura di}\\
	Davide Ferracin
}
\date{}

\begin{document}
\maketitle

\null % Serve scrivere qualcosa prima di \vfill altrimenti non funziona
\vfill % Riempie lo spazio verticale della pagina
\hspace*{-1.5em}\includegraphics[scale=.7]{by-sa-icon.pdf}
\begin{flushleft}
	Quest'opera è stata rilasciata con licenza \emph{Creative Commons} Attribuzione - Condividi allo stesso modo 4.0 Internazionale. Per leggere una copia della licenza visita il sito web \url{http://creativecommons.org/licenses/by-sa/4.0/}.\\
	2015, Davide Ferracin (\href{mailto:davide.ferracin@studenti.unimi.it}{\ttfamily davide.ferracin@studenti.unimi.it})\\[1cm]
	Versione aggiornata al \today.
\end{flushleft}

\tableofcontents
\chapter{Gruppi e anelli}
\section{Gruppi} \label{sec:gruppi}
\begin{definizione} \label{d:gruppo}
Si definisce \emph{gruppo} un insieme $G$ non vuoto munito di un'operazione binaria interna $*\colon G\times G\to G$, ossia tale per cui sono rispettati gli assiomi seguenti:
\begin{enumerate}
\item vale la proprietà associativa, cioè $\forall g_1,g_2,g_3\in G$ vale $g_1*(g_2*g_3)=(g_1*g_2)*g_3$;
\item esiste l'elemento neutro, cioè $\exists e\in G\colon\forall g\in G$, $g*e=e*g=g$;
\item esiste l'inverso, ossia $\forall g\in G$ $\exists g'\in G\colon g*g'=g'*g=e$.
\end{enumerate}
\end{definizione}
Dove non ci saranno ambiguità, d'ora in poi indicheremo l'operazione interna del gruppo come una moltiplicazione, omettendo il simbolo $*$: scriveremo dunque $x*y=xy$. L'inverso di un elemento $x$ sarà indicato, coerentemente, con $x^{-1}$.

La struttura di gruppo si compone sempre di un insieme e di un'operazione, perciò si identifica convenzionalmente con la coppia $(G,*)$; uno insieme può formare gruppi differenti in base all'operazione associata.
Il gruppo è detto \emph{commutativo} (o \emph{abeliano}) se vale anche la proprietà commutativa, cioè $\forall x,y\in G$, $xy=yx$.

L'elemento neutro di un gruppo è sempre unico: se $e$ ed $e'$ rispettano la seconda proprietà, allora $e'=ee'=e$ quindi coincidono.
Lo stesso vale per l'inverso, dato $x\in G$: se $a$ e $b$ sono due inversi di $x$, allora
\begin{equation}
	b=eb=(ax)b=a(xb)=ae=a.
	\label{eq:unicita-inverso}
\end{equation}

Elenchiamo di seguito alcuni esempi, più o meno immediati, di gruppi.
\begin{itemize}
	\item Gli insiemi $\Z$, $\Q$, $\R$, $\C$ con l'usuale operazione di addizione. Più in generale, gli elementi di un campo qualsiasi formano un gruppo rispetto all'addizione.
	\item Gli insiemi $\Q$, $\R$, e $\C$, privati dello zero, con l'usuale moltiplicazione. Lo stesso accade per gli elementi non nulli di un campo qualsiasi: dato un campo $K$, saremo soliti indicare il gruppo moltiplicativo $(K\setminus\{0\},\cdot)$ con il simbolo $K^\times$.
	\item Le rotazioni in un piano, con l'operazione di composizione, che è anche abeliano.
	\item Le rotazioni in $\R^3$, sempre con la composizione, sono ancora un gruppo. Esso però non è abeliano, perch\'e due rotazioni effettuate rispetto ad assi differenti in generale non commutano.
	\item L'insieme $\{-1,1\}$ forma un gruppo rispetto alla moltiplicazione.
\end{itemize}

\begin{definizione} \label{d:sottogruppo}
	Un sottoinsieme $H$ di un gruppo $G$ è detto \emph{sottogruppo} di $G$ se è a sua volta un gruppo con l'operazione di $G$.
\end{definizione}
In altre parole, $H$ è un gruppo contenuto in un gruppo più grande.
Questa definizione equivale alle richieste:
\begin{enumerate}
	\item $H$ deve essere chiuso rispetto all'operazione del gruppo $G$, ossia se $a,b\in H$ allora $ab,ba\in H$;
	\item ogni elemento di $H$ deve avere il suo inverso in $H$.
\end{enumerate}
Di conseguenza, $H$ deve anche contenere l'elemento neutro (lo stesso!) di $G$, poich\'e se $x\in H$, allora anche $x^{-1}$ vi appartiene, dunque anche $xx^{-1}=e$.

Ogni gruppo ammette sempre due sottogruppi: il gruppo stesso e il \emph{sottogruppo banale} $\{e\}$ del suo elemento neutro.
Gli altri sottogruppi, se esistono, sono detti \emph{propri}.
Se un gruppo è abeliano, allora anche tutti i suoi sottogruppi lo sono.
Alcuni esempi di sottogruppi sono i seguenti.
\begin{itemize}
	\item L'insieme $\T=\{z\in\C\colon \abs{z}=1\}$, detto \emph{gruppo circolare}, è un sottogruppo di $\C^\times$.
		Poich\'e $\C^\times$ è abeliano, lo è anche $\T$.
	\item In $\R^2$, sia $R(\theta)$ la rotazione antioraria di un angolo $\theta$.
		L'insieme $\{R(0), R(\pi/2), R(\pi), R(3\pi/2)\}$ è un sottogruppo del gruppo delle rotazioni nel piano.
\end{itemize}

\section{Relazioni di equivalenza} \label{sec:relazioni-equivalenza}
Una relazione binaria su un insieme $X$ lega due elementi dell'insieme.
Essa si definisce come un sottoinsieme di $X\times X$, intendendo che due elementi $a,b$ sono messi in relazione da $R$ se $(a,b)\in R$.
Solitamente una relazione di questo tipo si indica con il simbolo $\sim$, cioè $a\sim b$.
\begin{definizione} \label{d:relazione-equivalenza}
	La relazione $\sim$ è una relazione di equivalenza su $X$ se è binaria e valgono le seguenti proprietà:
	\begin{enumerate}
		\item è riflessiva: $a\sim a$;
		\item è simmetrica: se $a\sim b$, allora $b\sim a$;
		\item è transitiva: se $a\sim b$ e $b\sim c$, allora anche $a\sim c$.
	\end{enumerate}
\end{definizione}
Con questa relazione di equivalenza possiamo ``raggruppare'' gli elementi di $X$ in vari insiemi di elementi tutti in relazione tra loro.
Vediamo se questa suddivisione è buona, cioè se un elemento è categorizzato in uno solo di questi insiemi o meno.
\begin{definizione}
	Si chiama \emph{classe di equivalenza} di un elemento $a\in X$, rispetto alla relazione $\sim$, l'insieme
	\begin{equation*}
		[a]=\{b\in X\colon b\sim a\}.
	\end{equation*}
	L'elemento $a$ è detto \emph{rappresentante} della classe $[a]$.
\end{definizione}
\begin{teorema}
	Due classi di equivalenza, rispetto alla relazione $\sim$, $[a]$ e $[a']$ coincidono se e solo se $a\sim a'$.
\end{teorema}
\begin{proof}
	Siano le due classi $[a]=\{x\in X\colon x\sim a\}$ e $[a']=\{y\in X\colon y\sim a'\}$.
	Sia $[a]\subseteq[a']$: preso un elemento $x\in[a]$, si ha ovviamente che $x\sim a$.
	Poiché $a\sim a'$, per la proprietà transitiva $x\sim a'$ quindi $x\in[a']$, e viceversa per simmetria: allora $[a]\equiv[a']$.
	Poniamo ora le due classi coincidenti, siccome $a\in[a]$ e $a'\in[a']$, poichè le due classi coincidono si ha che $a$ appartiene anche ad $[a']$ e $a'$ appartiene anche ad $[a]$, quindi $a\sim a'$.
\end{proof}

\begin{teorema}
	Due classi di equivalenza sono distinte se e solo se sono disgiunte: se $[a]\neq[b]$ allora $[a]\cap[b]=\emptyset$.
\end{teorema}
\begin{proof}
	Sia per assurdo che esista un elemento $c\in[a]\cap[b]$.
	Allora esso è in relazione sia con $a$ che con $b$, ma allora per la proprietà transitiva $a\sim b$, quindi le due classi coincidono, il che è una contraddizione.
	Le due classi devono quindi essere disgiunte.
\end{proof}

Le classi distinte individuate da una relazione di equivalenza in $X$ costituiscono una partizione di $X$.
\begin{definizione} \label{d:partizione}
	Sia $\{S_i\}_{i\in I}$ una famiglia di sottoinsiemi di un insieme $X$. Tale famiglia si dice \emph{partizione} di $X$ se:
	\begin{itemize}
		\item $S_i\neq\emptyset$ $\forall i\in I$;
		\item $S_i\cap S_j=\emptyset$ per ogni $i\neq j$;
		\item $\bigcup_{i\in I}S_i\equiv X$.
	\end{itemize}
\end{definizione}
Sia $\{S_i\}_{i\in I}$ una partizione di un insieme $X$: si può sempre definire una relazione di equivalenza $\sim$ su $X$, ponendo che $\forall a,b\in X$, $a\sim b$ se e solo se $\exists i\in I\colon a,b\in S_i$.
Una tale relazione soddisfa la definizione di relazione di equivalenza:
\begin{enumerate}
	\item Qualsiasi $a\in X$ sta in almeno uno dei sottoinsiemi $S_i$, per il terzo punto della \ref{d:partizione}, quindi $a\sim a$.
	\item Se esiste un $i\in I$ per cui $a,b\in S_i$, certamente scambiando l'ordine di $b$ e $a$ entrambi appartengono comunque a $S_i$, quindi se $a\sim b$ anche $b\sim a$.
	\item Se $a\sim b$ e $b\sim c$, allora esiste $i\in I$ per il quale $a,b\in S_i$ ed esiste un altro indice $j\in I$ per cui $b,c\in S_j$.
		Se $i\neq j$, però, $b$ non potrebbe appartenere ad entrambi perché la loro intersezione sarebbe vuota.
		Allora $i=j$, e per tale indice $a,b,c\in S_i$ (o $S_j$), quindi $a\sim c$.
\end{enumerate}



\section{Anelli} \label{sec:anelli}
\begin{definizione} \label{d:anello}
Un insieme non vuoto $A$, dotato di due operazioni binarie interne $*$ e $\diamond$, si dice anello se valgono le seguenti proprietà:
\begin{enumerate}
\item $(A,*)$ è un gruppo abeliano;
\item $(A,\diamond)$ è un semigruppo, cioè è solo associativo;
\item $\forall a,b,c\in A$ valgono $(a* b)\diamond c=(a\diamond c)*(b\diamond c)$ e $a\diamond(b* c)=(a\diamond b)*(a\diamond c)$.
\end{enumerate}
\end{definizione}
Intenderemo sempre che la seconda operazione avrà sempre la precedenza sulla prima, se non diversamente specificato: vale a dire, $x* y\diamond z$ significherà $x*(y\diamond z)$; in caso contrario si usano le parentesi dove necessario.
D'ora in poi, per mantenere una notazione più familiare e semplice, ci riferiremo all'operazione $*$ come ad un'\emph{addizione} (e la indicheremo con $+$), e all'operazione $\diamond$ come ad una \emph{moltiplicazione}.
Infatti l'addizione e la moltiplicazione che tutti conosciamo soddisfano questi assiomi, che comunque possono essere generalizzati ad operazioni differenti, come il prodotto tra polinomi o matrici.

L'anello si dice \emph{commutativo} se anche $(A,\cdot)$ è commutativo, cioè $ab=ba$ $\forall a,b\in A$; la commutatività dell'addizione è sempre garantita dal fatto che $(A,+)$ è abeliano.
L'elemento neutro dell'addizione in un anello esiste sempre, dato che $(A,+)$ è un gruppo: indicheremo tale elemento con $0$, o con $0_A$ se ci sarà bisogno di specificare l'anello al quale appartiene.
L'esistenza dell'elemento neutro della moltiplicazione, invece, non è data per certa: se esiste, l'anello si dice \emph{dotato di unità}, e la indicheremo con $1$ o $1_A$.

Ecco alcuni esempi di anelli.
\begin{itemize}
\item $\Z$ è un anello con le usuali operazioni di addizione e moltiplicazione, come del resto $\Q$, $\R$, $\C$ e ogni altro campo.
%\item Fissato un numero $n$ intero e non nullo, si stabilisce la relazione $\sim$ definita come $a\sim b\iff n\divides a-b$, ossia $n$ divide la differenza $a-b$. Si scrive anche che $a\equiv b\mod n$.
%Si dimostra che $\sim$ così definita è una relazione di equivalenza e congruenza.
%L'insieme delle classi di equivalenza distinte, indicato con $\Z_n$, è un anello rispetto alle operazioni di somma e prodotto definite come segue: $\forall [a]_n,[b]_n\in\Z_n$, $[a]_n+[b]_n=[a+b]_n$ e $[a]_n[b]_n=[ab]_n$. La classe $[1]_n$ è l'unità (per qualsiasi $n$) per il prodotto; inoltre, $[a]_n\equiv[b]_n$ se e solo se $a\sim b$ cioè $b=a+kn$, quindi $[a]_n=\{a+kn,\ k\in\Z\}$.
\item L'insieme $K[x]$ dei polinomi (di grado qualunque) con termini presi da un campo $K$ forma un anello con le note operazioni di somma e prodotto tra polinomi.
\end{itemize}

Definiamo per $n\in\Z$ l'addizione di $a$ con se stesso $n-1$ volte come
\begin{equation}
	na=\underbracket[.5pt]{a+a+\cdots+a}_\textup{$n$ volte},
\end{equation}
per $n\in\N$, e con $0a=0_A$; per $n<0$, basta porre $na=-(-n)a$ per ricondursi ai casi precedenti.
Definiamo poi $a$ moltiplicato con se stesso $n-1$ volte come
\begin{equation}
	a^n=\underbracket[.5pt]{a\cdot a\cdot a\cdots a}_\textup{$n$ volte}
\end{equation}
con $n>0$, e (se esiste l'unità) $a^0=1_A$.
Per $n<0$ questa operazione non è definita.

Ricaviamo alcune semplici proprietà delle due operazioni.
\begin{itemize}
	\item $0_Aa=a0_A=0_A$.
		Possiamo infatti scrivere sempre $0_Aa=(0_A+0_A)a$, e per la proprietà distributiva abbiamo $0_Aa=0_Aa+0_Aa$, per cui aggiungendo l'opposto $-0_Aa$ ai due membri (esiste sempre, essendo $(A,+)$ un gruppo) troviamo $0_aa=0$.
		La dimostrazione è analoga per $a0_A$.
	\item $(na)b=a(nb)=n(ab)$, che di dimostra facilmente sfruttando la proprietà distributiva partendo da $(a+a+\dots+a)b$.
	\item $a(-b)=(-a)b=-(ab)$, ponendo $n=-1$ nella precedente.
\end{itemize}

\begin{definizione} \label{d:divisore-zero}
Un elemento $a\neq 0_A$ di un anello $A$ si dice \emph{divisore dello zero} se esiste un elemento $b\in A$ tale che $ab=0_A$ oppure $ba=0_A$.
\end{definizione}
Ovviamente i due casi coincidono se l'anello è commutativo, ma in generale non lo si può affermare.
\paragraph{Esempi}
\begin{itemize}
	\item L'insieme $\cont{}(-1,1)$ delle funzioni $f\colon(-1,1)\to\R$ continue è un anello con addizione e moltiplicazione.
			In esso, definiamo le funzioni
			\begin{equation*}
				f(x)=
				\begin{cases}
					0& -1<x<0\\
					x^2&0\le x<1
				\end{cases}\qquad\text{e}\qquad
				g(x)=
				\begin{cases}
					x^2& -1<x<0\\
					0&0\le x<1
				\end{cases}
			\end{equation*}
			La $f$ è un divisore dello zero, in quanto $g$ non è la funzione identicamente nulla di $\cont{}(-1,1)$, ma $fg=0$ per ogni $x\in(-1,1)$.
			Per lo stesso motivo, ovviamente, anche $g$ è divisore dello zero.
\begin{comment}
		\item Nell'anello $\mat_{2,2}(\R)$, la matrice $\begin{psmallmatrix}1&0\\0&0\end{psmallmatrix}$ è un divisore dello zero perché
		\begin{equation*}
			\begin{pmatrix}1&0\\0&0\end{pmatrix}\begin{pmatrix}0&0\\1&0\end{pmatrix}=\begin{pmatrix}0&0\\0&0\end{pmatrix}.
		\end{equation*}
	Tale proprietà può essere estesa alle matrici $2\times 2$ in un campo $K$ generico, utilizzando l'unità e lo zero $1_K$ e $0_K$.
		\item In $\Z_{10}$, si prendano le classi $[2]_{10}$ e $[5]_{10}$. Entrambe non sono nulle, perché 2 e 5 non sono divisibili per 10 (non valgono le relazioni $2\equiv 0\mod 10$ e $5\equiv 0\mod 10$). Moltiplicandole, però, risulta $[2]_{10}[5]_{10}=[10]_{10}=[0]_{10}$, perché ovviamente 10 è divisibile per se stesso, quindi $[2]_{10}$ è un divisore dello zero in $\Z_{10}$. Per la commutatività dell'anello, anche $[5]_{10}$ lo è.
	Più in generale, in un anello $\Z_n$ sono divisori dello zero le classi $[a]_n$ dove $a$ è un divisore non banale di $n$.
\end{comment}
\end{itemize}

\begin{teorema}
	Un anello $A$ è privo di divisori dello zero se e solo se valgono le leggi di cancellazione per il prodotto.\footnote{Ossia se per ogni $a,x,y\in A$ con $a\ne 0$ le relazioni $ax=ay$ e $xa=ya$ implicano $x=y$.}
\end{teorema}
\begin{proof}
	Supponiamo che $A$ sia un anello privo di divisori dello zero, e prendiamo l'ipotesi $ax=ay$: dalla proprietà distributiva si ha $a(x-y)=0$.
	Dato che non esistono divisori dello zero in $A$, se $a\ne 0$ deve necessariamente essere $x-y=0$, ossia $x=y$.
	La dimostrazione per $xa=ya\then x=y$ è del tutto analoga.
	
	Partiamo ora dalle relazioni $ax=ay\then x=y$ e $xa=ya\then x=y$.
	Se esistessero $x,y\ne 0$ tali che $xy=0$ (ossia $x$ e $y$ divisori dello zero), allora risulterebbe anche $xy=0=x0$ da cui $y=0$, poich\'e valgono le leggi di cancellazione del prodotto.
	Ma ciò contraddice l'ipotesi che $x,y\ne 0$ quindi tali $x$ e $y$ non possono esistere: allora $A$ è privo di divisori dello zero.
\end{proof}

\begin{definizione} \label{d:elemento-invertibile}
	Sia $A$ un anello con unità.
	Un elemento $a\in A$ si dice \emph{invertibile} se esiste $b\in A$ tale per cui $ab=ba=1$.
	Tale $b$ si indica con $a^{-1}$.
\end{definizione}

\begin{teorema}
	Se $A$ è un anello dotato di unità, i suoi elementi invertibili non sono divisori dello zero.
\end{teorema}
\begin{proof}
	Se esistesse un elemento $a\in A$ invertibile e divisore dello zero, allora esisterebbe un elemento $b\in A\setminus\{0\}$ tale che $ab=0$.
	Si ottiene però che
	\begin{equation}
		b = 1b = a^{-1}ab = a^{-1}0 = 0
	\end{equation}
	ossia $b=0$, che contraddice l'ipotesi $b\ne 0$ legata all'esistenza di $b$.
	Dunque non può esistere un tale $b$: vale a dire, $a$ non è un divisore dello zero.
\end{proof}

\begin{definizione} \label{d:dominio-integrita}
	Un anello si dice \emph{dominio d'integrità} se è commutativo ed è privo di divisori dello zero.
\end{definizione}
Sono domini d'integrità gli anelli di $\Z$, $\Q$, $\R$, $\C$ con le usuali operazioni di somma e prodotto.

\begin{definizione}
Si chiama \emph{corpo} un anello dotato di unità in cui ogni elemento non nullo è invertibile.
\end{definizione}

\begin{definizione} \label{d:campo1}
Si dice \emph{campo} un corpo commutativo con almeno due elementi.
\end{definizione}
Il piccolo teorema di Wedderburn afferma, inoltre, che ogni corpo finito è un campo.
Possiamo dare una definizione alternativa di campo, equivalente alla precedente, basata su degli assiomi.
\begin{definizione} \label{d:campo2}
Si definisce \emph{campo} la terna $(K,+,\cdot)$ in cui $K$ è un insieme non vuoto e $+$ e $\cdot$ sono operazioni interne $K\times K\to K$, per le quali:
\begin{itemize}
\item $(K,+)$ è un gruppo abeliano, con elemento neutro $0_K$;
\item $(K\setminus\{0_K\},\cdot)$ è un gruppo abeliano, con elemento neutro $1_K$;
\item vale la proprietà distributiva, per cui $\forall a,b,c\in K$ vale $a\cdot(b+c)=a\cdot b+a\cdot c$.
\end{itemize}
\end{definizione}
Questa definizione è in sostanza una generalizzazione della struttura di $(\R,+,\cdot)$. È quindi, ovviamente, un campo $(\R,+,\cdot)$, e lo sono anche $(\Q,+,\cdot)$ e $(\C,+,\cdot)$.


\section{Ideali} \label{sec:ideali}
\begin{definizione} \label{d:ideale}
	Sia $A$ un anello e $I$ un suo sottoinsieme.
	Se $(I,+)$ è un sottogruppo di $(A,+)$ e per ogni $x\in I$ e $a\in A$:
	\begin{itemize}
		\item $ax\in I$, allora $I$ è detto \emph{ideale sinistro};
		\item $xa\in I$, allora $I$ è detto \emph{ideale destro};
		\item $ax,xa\in I$, allora $I$ è detto \emph{ideale bilatero}.
	\end{itemize}
\end{definizione}
In altre parole, preso un elemento di un ideale sinistro (destro) possiamo moltiplicarlo a sinistra (destra) per qualsiasi elemento dell'anello e ottenere ancora un elemento dell'ideale.
Nel caso di un anello commutativo, le tre definizioni naturalmente coincidono, e parleremo semplicemente di \emph{ideale}.

Dalla chiusura di $I$ rispetto alla somma otteniamo inoltre che qualsiasi ideale deve sempre contenere lo zero dell'anello.
Ogni anello ammette sempre due ideali detti \emph{banali}: $\{0\}$ e l'anello $A$ stesso.
Gli ideali non banali sono detti \emph{propri}.
Se l'anello è dotato di unità, allora un ideale è proprio se e solo se non la contiene: se infatti $I\ni 1$, allora poich\'e il prodotto $1a$ per qualsiasi $a$ nell'intero anello deve essere incluso in $I$, tale ideale contiene tutti gli elementi di $A$, ma allora $I=A$.

Degli esempi importanti di ideali, che incontreremo in seguito, sono i seguenti.
\begin{itemize}
	\item Dato $p\in\Z$, chiamiamo $p\Z$ l'insieme $\{x\in\Z\colon x=np, n\in\Z\}$ ossia l'insieme dei multipli interi di un certo intero $p$.
		Se $x,y\in p\Z$, siano essi $x=np$ e $y=mp$, allora $x+y=(n+m)p$ e poich\'e $n+m\in\Z$ allora $x+y\in p\Z$.
		Per un qualunque $z\in\Z$, inoltre, $xz=npz=(nz)p$ e $nz\in\Z$ quindi $xz\in p\Z$.
		L'insieme $p\Z$ è dunque un ideale di $\Z$, per qualunque $p$.
	\item I numeri pari formano l'insieme $2\Z$, che è un ideale per il punto precedente.
		I numeri dispari, invece, non formano un ideale, in quanto non comprendono lo zero.
	\item Nell'anello delle funzioni continue $\cont{}(\R)$, è un ideale l'insieme delle funzioni che si annullano in un dato punto, ad esempio per cui $f(1)=0$.
\end{itemize}

Definiamo ora alcuni tipi particolari di ideali (per semplicità, le daremo per anelli commutativi).
\begin{definizione} \label{d:ideale-primo}
	Un ideale $I$ di un anello $A$ è detto \emph{primo} se:
	\begin{itemize}
		\item è un sottoinsieme proprio di $A$;
		\item se $a,b\in A$ sono tali che $ab\in I$, allora almeno uno dei due appartiene a $I$.
	\end{itemize}
\end{definizione}
Questo concetto ricalca la definizione di \emph{numeri primi}: se un numero $p\in\Z$ è primo, ogni volta che divide un prodotto $xy$ con $x,y\in\Z$ allora $p$ divide $a$ oppure $b$.
Le due definizioni sono in effetti collegate: un numero intero (positivo) $n$ è primo se e solo se $n\Z$ è un ideale primo.
 
\begin{definizione} \label{d:ideale-massimale}
	Un ideale $A$ si dice \emph{massimale} se $I\neq A$, per ogni ideale $J\supseteq I$ si ha o che $J=I$ oppure $J=A$.
\end{definizione}
Gli ideali massimali sono dunque degli elementi massimali rispetto all'operazione di inclusione insiemistica tra gli \emph{ideali propri} di un anello (escludendo quindi dalle opzioni l'anello stesso).
Essi non sono contenuti propriamente in nessun altro ideale proprio dell'anello.

Se ogni elemento $x$ di un ideale $I$, di un anello $A$, può essere scritto come
\begin{equation*}
	x=\sum_{i=1}^na_ki_k
\end{equation*}
con $a_k\in A$ e $i_k\in I$, ossia come combinazione lineare di un numero finito di suoi elementi $\{i_k\}_{k=1}^n$, diciamo che l'ideale è \emph{generato} da tali elementi, e si indica solitamente come $I=(i_1,\dots,i_n)$.
Il caso in cui l'ideale è generato da un solo elemento è di particolare importanza, e merita una sua definizione.
\begin{definizione} \label{d:ideale-principale}
	Un ideale $I$ di un anello $A$ si dice \emph{principale} se è generato da un solo elemento.
\end{definizione}
In linea con la notazione precedente, l'ideale principale generato da $a$ si indica con $(a)$.
Si può dimostrare che l'ideale principale $(a)$ è il più piccolo ideale che comprende $a$.

\section{Anelli quoziente} \label{sec:anelli-quoziente}
Riprendiamo ora le relazioni di equivalenza, introdotte nel capitolo \ref{sec:relazioni-equivalenza}.
Dati un ideale (bilatero) $I$ di un anello $A$ e due elementi $a,b\in A$, stabiliamo la relazione
\begin{equation*}
	a\sim b\iff a-b\in I.
\end{equation*}
È facile vedere, con le proprietà degli ideali, che tale relazione è anche di congruenza.
Se $a\sim b$ si dice anche che $a$ e $b$ sono \emph{congruenti modulo $I$}.
Da essa possiamo costruire le classi di equivalenza nell'anello: la classe $[a]_I$ (indichiamo con il pedice $[\cdot]_I$ il fatto che la relazione è basata sull'ideale $I$, per maggiore chiarezza) consiste in tutti quegli elementi $x$ di $A$ che ``distano $a$ dall'ideale $I$'', ossia tali per cui $x-a\in I$.
Alternativamente, gli elementi $x\in[a]_I$ sono la somma di $a$ e di un elemento dell'ideale $I$, e per questo motivo si indica la classe di equivalenza come $I+a$.
Formalmente, dunque,
\begin{equation*}
	[a]_I=I+a=\{x\in A\colon x=a+i, i\in I\}.
\end{equation*}
Possiamo definire delle operazioni su queste classi come di seguito:
\begin{itemize}
	\item l'addizione di $[a]_I=I+a$ e $[b]_I=I+b$ come la classe di rappresentante $a+b$, ossia $I+(a+b)$;
	\item analogamente, la moltiplicazione di due classi $[a]_I[b]_I=(I+a)(I+b)$ come la classe che ha come rappresentante il prodotto dei due rappresentanti, ossia $I+ab$.
\end{itemize}
Si può verificare che queste operazioni sono ben definite, ossia che non dipendono dalla scelta dei rappresentanti.
Con queste due operazioni, l'insieme delle classi di equivalenza forma un anello, detto \emph{anello quoziente} (rispetto alla relazione stabilita).

\begin{definizione} \label{d:anello-quoziente}
	Dato un ideale bilatero $I$ di un anello $A$, si chiama \emph{anello quoziente} l'insieme, indicato con $A\quot I$, delle classi di equivalenza $[a]_I=\{x\in A\colon x=a+i, i\in I\}$, con le operazioni
	\begin{equation}
		\begin{gathered}
			(I+a)+(I+b)=I+(a+b)\\ (I+a)(I+b)=I+ab.
		\end{gathered}
		\label{eq:operazioni-anello-quoziente}
	\end{equation}
\end{definizione}
Lo zero dell'anello quoziente è indicato come $I+0$, ed è chiaramente l'ideale $I$ stesso.
Notiamo che se $a\in I$, allora $I+a$ è ancora lo zero di $A\quot I$: infatti, essendo $I$ chiuso rispetto alla somma, l'addizione di un elemento dell'ideale (cioè $I$) con $a$ (che è in $I$) produce ancora un elemento nell'ideale, vale a dire un elemento di $I=I+0$.
L'identità moltiplicativa, se esiste, sarà indicata con $I+1$.

Proviamo a prendere il quoziente di $A$ con gli ideali banali.
\begin{itemize}
	\item Per $I=\{0\}$, scelto un $a\in A$ abbiamo che $b\in [a]=I+a$ se $b-a\in I$, cioè $b-a=0$: ma ciò è possibile solo se $b=a$, dunque $I+a=\{a\}$ per qualsiasi $a\in A$.
	\item Per $I=A$, se $b\in I+a$ dovrà essere $b-a\in A$: questo è sempre vero qualsiasi sia $b$, quindi $I+a=A$ per qualsiasi $a$!
		Ciò significa che $A\quot A$ è composto da un solo elemento.
\end{itemize}

L'ideale $I=2\Z$ di $\Z$ è massimale: infatti se un ideale $J$ contiene $I$, allora $J=k\Z$ per un $k\in N$ che sia divisore di 2.
Ma allora $k\in\{1,2\}$, cioè $k\Z$ è ancora $2\Z$ oppure è tutto $\Z$.
Perciò $2\Z$ è un ideale massimale; lo stesso di dimostra per qualsiasi $p\Z$ con $p$ primo.
Questo risultato si generalizza nel seguente teorema.
\begin{teorema} \label{t:ideale-primo-quoziente-integro}
	Sia $A$ un anello commutativo con unità e sia $I\subset A$ un ideale proprio: allora $I$ è primo se e solo se $A/I$ è un dominio d'integrità.
\end{teorema}
\begin{proof}
	Supponiamo che $A\quot I$ sia un dominio di integrità, per cui prese due classi $I+a$ e $I+b$, se $I+ab=I+0$ deve necessariamente risultare $I+a=I+0$ oppure $I+b=I+0$.
	Passando dalle classi di $A\quot I$ agli elementi di $A$, il fatto che $I+ab$ sia $I+0$ significa che $ab$ è nell'ideale $I$.
	Analogamente se $I+a=I+0$ significa che $a\in I$.
	Ma ciò vuol dire che se $ab\in I$ allora uno dei due tra $a$ e $b$ è necessariamente nell'ideale: questa è proprio la definizione di ideale primo, quindi $I$ è primo.

	Sia ora $I$ un ideale primo: se $ab\in I$, almeno uno tra $a$ e $b$ deve appartenere a $I$.
	Nel linguaggio delle classi di equivalenza ciò significa che se $(I+a)(I+b)=I+ab=I+0$, allora $I+a=I+0$ oppure $I+b=0$.
	Queste affermazioni sono equivalenti a dire che non esistono $a,b\notin I$ tali che $ab\in I$, cioè
	\begin{equation*}
		\nexists I+a, I+b\in A\quot I\colon (I+a)(I+b)=I+0
	\end{equation*}
	quindi $A\quot I$ è un dominio di integrità.
\end{proof}

\begin{teorema} \label{t:ideale-massimale-quoziente-campo}
	Sia $A$ un anello commutativo con unità e sia $I\subset A$ un ideale proprio: $I$ è massimale se e solo se $A/I$ è un campo.
\end{teorema}
\begin{proof}
	Sia $I$ un ideale massimale: allora non può esistere un ideale $J$ tale che $I\subset J\subset A$.
	Dimostriamo che $A\quot I$ è un campo mostrando che ogni suo elemento non nullo è invertibile.
	Sia $I+a$ un elemento non nullo di $A\quot I$, ossia deve essere $a\notin I$.
	Fissato questo elemento, costruiamo l'insieme $J_a=\{j\in A\colon j=i+ax, i\in I, x\in A\}$.
	Sicuramente, poich\'e $A$ è commutativo, lo è anche $J_a$.
	Inoltre è anche un ideale: infatti, dati $j_1=i_1+ax_1$, $j_2=i_2+ax_2$ e $b\in A$ abbiamo
	\begin{equation}
		\begin{gathered}
			j_i+j_2=\underbracket[.5pt]{i_1+i_2}_{\in I}+a(\underbracket[.5pt]{x_1+x_2}_{\in A})\in J_a;\\
			j_1b=(i_1+ax_1)b=\underbracket[.5pt]{i_1b}_{\in I}+a(\underbracket[.5pt]{x_1b}_{\in A})\in J_a.
		\end{gathered}
	\end{equation}
	Tutti gli elementi $i\in I$ sono della forma $i+a0$, dunque $I\subseteq J$.
	Esistono anche elementi di $J$ che non appartengono a $I$?
	Dato che $I$ è proprio, non può contenere l'unità, come avevamo già visto.
	Allora l'elemento $i+a1=i+a$ appartiene a $J$, ma non a $I$.\footnote{Se $i+a\in I$, ossia $i+a=i'$ per qualche $i'\in I$, allora seguirebbe che $a=i'-i$, cioè $a\in I$.}
	Di conseguenza $I\subset J$: per la massimalità di $I$, però, ciò implica $J\equiv A$.
	Perciò $J$ deve contenere l'unità, che potremo dunque scrivere come $i^*+ax^*$ per qualche $i^*\in I$ e $x^*\in A$.
	Preso questo $x^*$, vediamo che la classe $I+x^*\in A\quot I$ è l'inverso di $I+a$:
	\begin{equation}
		(I+x^*)(I+a)=I+ax^*=I+(1-i^*)=I+1
	\end{equation}
	poich\'e se $i^*+ax^*=1$ allora $ax^*=1-i^*$, e $I-i^*= I$.
	Ma $I+1$ è l'unità di $A\quot I$, dunque ogni elemento non nullo (ossia con $a\notin I$) di $A\quot I$ ammette un inverso: ciò prova che $A\quot I$ è un campo.

	Sia ora $A\quot I$ un campo: allora, poich\'e deve possedere almeno due elementi, $I$ non può essere uguale ad $A$, perch\'e come abbiamo già visto $A\quot A$ contiene un solo elemento.
	Dunque $I$ è un ideale proprio.
	Prendiamo ora un ideale $J$ tale che $I\subseteq J\subseteq A$ con $I\ne J$.
	Esiste dunque un $x$ che appartiene a $J$ ma non a $I$, di conseguenza $I+x\ne I+0$.
	Non essendo l'elemento nullo di $A\quot I$, che per ipotesi è un campo, $I+x$ è invertibile: esiste una classe $I+y\in A\quot I$ tale per cui
	\begin{equation*}
		I+1=(I+y)(I+x)=I+xy,
	\end{equation*}
	quindi $xy=1+i^*$ per qualche $i^*\in I$.
	Ora, $I\subseteq J$, perciò $i^*\in J$, e analogamente $x\in J$ quindi anche $xy\in J$: ma allora anche $1=xy-i^*$ appartiene a $J$.
	Dato che $J$ contiene l'unità, segue necessariamente che $J=A$, perciò $I$ è massimale.
\end{proof}
\begin{corollario} \label{c:ideale-massimale-primo}
	In un anello commutativo con unità, ogni ideale massimale è primo.
\end{corollario}
\begin{proof}
	Se $I$ è massimale, per il teorema \ref{t:ideale-massimale-quoziente-campo} $A\quot I$ è un campo, quindi in particolare è anche un dominio di integrità: ma allora dal teorema \ref{t:ideale-primo-quoziente-integro} $I$ è primo.
\end{proof}
L'implicazione inversa, ossia che ogni ideale primo è massimale, in generale è falsa (il problema sta nell'affermazione ``un campo è un dominio d'integrità'', che non si può invertire).
Vedremo in che ambito essa è vera quando introdurremo i domini \emph{a ideali principali}.

\section{Omomorfismi di anelli}
\begin{definizione} \label{d:omomorfismo-anelli}
	Siano $(A,+,\cdot)$ e $(B,*,\diamond)$ due anelli: un \emph{omomorfismo} di anelli è un'applicazione $\phi\colon A\to B$ che preserva le operazioni, cioè tale che per ogni $a,b\in A$ si ha 
	\begin{equation}
		\phi(a+b)=\phi(a)*\phi(b)\text{ e }\phi(ab)=\phi(a)\diamond\phi(b).
	\end{equation}
	Se gli anelli sono dotati di unità, si richiede che l'omomorfismo, oltre alle operazioni, preservi anche l'unità, ossia $\phi(1_A)=1_B$.
\end{definizione}
Ad esempio la funzione da $\Z$ in s\'e definita come $\phi(a)=0$ per qualsiasi $a\in\Z$, cioè che porta qualsiasi elemento nello zero, chiaramente preserva le operazioni, ma non è un omomorfismo d'anelli in quanto $\phi(1)=0$ che ovviamente non è l'unità di $\Z$.

\begin{definizione} \label{d:nucleo-omomorfismo-anelli}
	Sia $\psi\colon A\to B$ un omomorfismo di anelli. Si definisce \emph{nucleo} di $\psi$ e si denota con $\Ker\psi$ l'insieme
	\begin{equation*}
		\Ker\psi=\{a\in A\colon\psi(a)=0_B\},
	\end{equation*}
	ossia l'insieme degli elementi di $A$ che hanno lo zero di $B$ come immagine.
\end{definizione}

\begin{teorema} \label{t:nucleo-ideale}
	Se $\psi\colon A\to B$ è un omomorfismo di anelli, allora il suo nucleo è un ideale di $A$.
\end{teorema}
\begin{proof}
	Verifichiamo le proprietà di ideale: per $x,y\in\Ker\psi$ e $a\in A$, si ha
	\begin{equation}
		\psi(x+y)=\psi(x)+\psi(y)=0_B+0_B=0_B
	\end{equation}
	quindi $x+y\in\Ker\psi$, e
	\begin{equation}
		\psi(ax)=\psi(a)\psi(x)=\psi(a)0_B=0_B
	\end{equation}
	quindi anche $ax\in\Ker\psi$, e analogamente per $xa$.
	Allora $\Ker\psi$ è proprio un ideale di $A$.
\end{proof}
Come già visto negli omomorfismi tra gruppi, anche un omomorfismo tra gli anelli $A$ e $B$ è iniettivo se e solo se il suo nucleo è $\{0_A\}$.
Se l'anello è commutativo, i suoi ideali sono tutti anche nuclei di omomorfismi di anelli.

\section{Anelli dei polinomi}
Passiamo ora a trattare un tipo di anelli molto importante: gli anelli composti da polinomi.
\begin{definizione} \label{d:polinomio}
	Si dice \emph{polinomio} a coefficienti in un anello $A$ una successione di elementi di $A$ definitivamente nulla:
	\begin{equation*}
		p=(a_0,a_1,a_2,\dots,a_n,0,0,\dots).
	\end{equation*}
\end{definizione}
I polinomi si rappresentano anche, più comunemente, indicando il posto di ogni elemento della successione con delle potenze di un'incognita, come ad esempio
\begin{equation}
	p(x)=a_0+a_1x+a_2x^2+\dots+a_nx^n.
\end{equation}
Generalmente, indicheremo i polinomi semplicemente con delle lettere, senza usare la notazione ``funzionale'' $p(x)$ ma solo $p$.
Useremo $p(x)$ invece quando esprimeremo il polinomio tramite le potenze di $x$, per evitare di confonderlo con i termini noti.

L'ultimo coefficiente non nullo del polinomio, $a_n$, che nella scrittura predecente è il coefficiente assegnato alla potenza di grado massimo, si dice \emph{coefficiente direttivo}.
Il termine $a_0$ è invece il \emph{termine noto}.

Tra polinomi definiamo la somma come
\begin{multline}
	(a_0,a_1,a_2,\dots,a_n,0,0,\dots)+(b_0,b_1,b_2,\dots,b_m,0,0,\dots)=\\
	(a_0+b_0,a_1+b_1,a_2+b_2,\dots,a_m+b_m,a_{m+1}+0,a_{m+2}+0,\dots,a_n,0,0,\dots),
\end{multline}
dove in questo caso $n>m$, e il prodotto come il polinomio che ha come componente di posto $k$ il coefficiente
\begin{equation}
	c_k=\sum_{i=0}^ka_ib_{k-i}.
\end{equation}
Con queste due operazione è facile vedere che $A[x]$, l'insieme dei polinomi a coefficienti in $A$, è a sua volta un anello.
Il polinomio unità di $A[x]$ è il polinomio avente come primo coefficiente l'unità di $A$, $1_A$, e tutti i successivi nulli; il polinomio nullo è il polinomio con tutti i coefficienti nulli.

Possiamo definire un'applicazione lineare $j\colon A\to A[x]$ che porta un elemento di $A$ nel polinomio avente come termine noto tale elemento, ossia la mappa
\begin{equation*}
	j(a)=(a,0,0,\dots).
\end{equation*}
Tale applicazione è un omomorfismo di anelli, in quanto preserva le operazioni e l'unità: infatti dati $a,b\in A$ risulta
\begin{gather*}
	j(a+b)=(a+b,0,0,\dots)=(a,0,0,\dots)+(b,0,0,\dots)=j(a)+j(b)\\
	j(ab)=(ab,0,0,\dots)=(a,0,0,\dots)(b,0,0,\dots)=j(a)j(b),
\end{gather*}
mentre porta l'unità $1_A$ nel polinomio $(1_A,0,0,\dots)$ che è l'unità di $A[x]$.
Si nota facilmente anche che tale omomorfismo $j$ è iniettivo, dato che $\Ker j=\{0_A\}$.

Ordinando gli elementi del polinomio in ordine di indici crescenti, la posizione del coefficiente direttivo indica il \emph{grado} del polinomio, che quando scritto come combinazione lineare di potenze è anche il grado della potenza massima che appare.

Un polinomio non nullo $p(x)=a_nx^n+\dots+a_0\in A[x]$, con $a_n\neq 0$, si dice di grado $n$, e si indica con $\deg p=\deg p(x)=n$.
Convenzionalmente si assegna al polinomio (identicamente) nullo il grado $-1$.
Se $a_n=1$, il polinomio si dice \emph{monico}.

\begin{proprieta} \label{pr:gradi-polinomi-operazioni}
Si hanno le seguenti proprietà tra i gradi dei polinomi e le operazioni:
\begin{itemize}
	\item $\deg(a+b)\leq\max\{\deg a,\deg b\}$;
	\item $\deg(ab)\leq\deg a+\deg b$.\footnote{Contrariamente a ciò che ci si aspetterebbe, non è un'uguaglianza perché pur essendo i coefficienti direttivi dei due polinomi non nulli, potrebbero esistere divisori dello zero in $A$, dunque non è detto che se $a,b\in A$ non sono nulli si abbia necessariamente $ab\neq 0$.}
\end{itemize}
\end{proprieta}

Ad esempio, nell'anello $\Z\quot 12\Z$, si considerino i due polinomi $a(x)=[6]x^2$ e $b(x)=[2]x$: si ha che $\deg a=2$ e $\deg b=1$.
La loro somma è un polinomio di grado 2, ma per il prodotto si vede che sebbene nell'anello $[6]$ e $[2]$ non siano nulli, lo è il loro prodotto perché $12$ appartiene alla stessa classe di equivalenza di $0$, quindi $a(x)b(x)=[6][2]x^3=[12]x^3=[0]x^3=[0]$; di conseguenza $\deg\big(ab\big)=-1\neq\deg a+\deg b=3$.
Le due proprietà precedenti sono in ogni caso rispettate.

Il caso dell'uguaglianza accade quindi soltanto se non esistono divisori dello zero nell'anello, vale a dire che esso è un dominio d'integrità: in questo caso il prodotto di due elementi non nulli non è mai nullo.

Siano $A$ e $B$ due anelli con unità, e $f\colon A\to B$ un omomorfismo d'anelli con unità.
Anche $\tilde{f}\colon A[x]\to B[x]$ definito come $\tilde{f}\big(p(x)\big)=f(a_n)x^n+\dots+f(a_0)$, con $f(a_i)\in B$, allora, è un omomorfismo di anelli con unità.
Non è detto però che sia $\deg\tilde{f}(p)=\deg p$: potrebbe accadere infatti che il coefficiente direttivo di $p$ appartenga al nucleo di $f$.
Infine, come richiesto dalla definizione, $\tilde{f}(1_{A[x]})=\big(f(1_A),0,0,\dots\big)=(1_B,0,0,\dots)=1_{B[x]}$.

\section{Divisione tra polinomi}
Sia $K[x]$ l'anello dei polinomi su un campo $K$: non avendo divisori dello zero, $ab\neq 0_K$ se $a$ e $b$ (elementi di $K$) non sono nulli.
Vale sempre, allora, l'uguaglianza $\deg(fg)=\deg f+\deg g$.

\begin{teorema}[Algoritmo euclideo delle divisioni successive]
	Sia $K$ un campo, e $a,b\in K[x]$, con $b$ diverso dal polinomio nullo.
	Esistono sempre, e sono unici, due polinomi $r$ e $q$ tali che
	\begin{equation}
		a=qb+r,\text{ con }\deg r<\deg b.
		\label{eq:algoritmo-divisione-polinomi}
	\end{equation}
\end{teorema}
\begin{proof}
	Dimostriamo l'esistenza dei due polinomi, per induzione, su $n=\deg a$.
	Se $n=-1$, allora $q=r=0$, cioè $qb+r=0$, e poiché per ipotesi $\deg b\ne -1$ perché non è nullo, si ha automaticamente che $\deg b>-1=\deg r$.
	I due polinomi cercati quindi sono entrambi dei polinomi identicamente nulli.
	Sia ora $n>-1$, e siano $a(x)=a_nx^n+\dots+a_0$, $b(x)=b_mx^m+\dots+b_0$.
	\begin{itemize}
		\item se $m>n$, allora poniamo $a=0\cdot b+a$ (scegliamo cioè $q=0$ nella \eqref{eq:algoritmo-divisione-polinomi}), e ciò significa che il resto è proprio $a$.
			Dunque $\deg r=n<m=\deg b$.
		\item se $m\leq n$, definiamo $\tilde{a}=\tilde{a}(x)=a(x)-b_m^{-1}a_nx^{n-m}b(x)$.
			Il coefficiente di grado massimo dell'ultimo termine è $-b_m^{-1}a_nx^{n-m}b_mx^m=-b_m^{-1}b_ma_nx^{n-m+m}=-a_nx^n$, quindi $\deg\tilde{a}\leq n-1$ poiché il coefficiente di grado $n$, che è il grado massimo di $a$, è cancellato.
		Per l'ipotesi di induzione, quindi, esistono due polinomi $\tilde{q}$ e $\tilde{r}$ tali che $\tilde{a}=\tilde{q}b+\tilde{r}$, per cui $\deg\tilde{r}<\deg b$.
		Risulta quindi
		\begin{align*}
			a(x)&=\tilde{q}(x)b(x)+b_m^{-1}a_nx^{n-m}b(x)+\tilde{r}(x)=\\
			&=\big(\tilde{q}(x)+b_m^{-1}a_nx^{n-m}\big)b(x)+\tilde{r}(x).
		\end{align*}
		Scelti $q(x)=\tilde{q}(x)+b_m^{-1}a_nx^{n-m}$ e $r(x)=\tilde{r}(x)$, si ha dunque l'uguaglianza $a=qb+r$ con $\deg r=\deg\tilde{r}<\deg b$.
	\end{itemize}
	Abbiamo trovato dunque che $q$ e $r$ secondo la \eqref{eq:algoritmo-divisione-polinomi} esistono sempre, qualunque sia il grado di $a$.

	Siano $q,r$ e $\overline{q},\overline{r}\in A[x]$ tali da soddisfare entrambe (le coppie) la \eqref{eq:algoritmo-divisione-polinomi}, ossia che $a=\overline{q}b+\overline{r}$ e $a=qb+r$, con $\deg\overline{r}<\deg b$ e $\deg r<\deg b$.
	Allora risulta
	\begin{equation}
		0=a-a=(q-\overline{q})b+r-\overline{r}
		\label{eqdim:unicita-quoziente-resto}
	\end{equation}
	Per la prima delle \ref{pr:gradi-polinomi-operazioni} risulta $\deg(r-\overline{r}\big)<\max\{\deg r,\deg\overline{r}\}<\deg b$.
	Se $\overline{q}\neq q$, poich\'e la loro differenza non sarebbe un polinomio nullo, si avrebbe $\deg(q-\overline{q})\ge 0$, e il prodotto $(\overline{q}-q)b$ darebbe un polinomio di grado sicuramente maggiore di quello di $b$, poiché $K$ non ha divisori dello zero: dunque $\deg\big((q-\overline{q})b\big)>\deg b$.
	Ma se dalla \eqref{eqdim:unicita-quoziente-resto} risulta $(\overline{q}-q)b=r-\overline{r}$, quindi i loro gradi devono essere uguali.
	Troviamo allora
	\begin{equation}
		\deg(r-\overline{r})=\deg\big((\overline{q}-q)b\big)>\deg b>\deg(r-\overline{r})
	\end{equation}
	che è chiaramente un assurdo.
	Di conseguenza deve essere $\overline{q}=q$ e $\overline{r}=r$, cioè i polinomi quoziente e resto sono unici.
\end{proof}
La dimostrazione di questo teorema è molto rigorosa, ma non è che una trascrizione ``più tecnica'' di quello che dovrebbe già essere noto dalla divisione tra polinomi (ma anche tra numeri naturali!):
\begin{itemize}
	\item Se $a$ è nullo, allora quoziente e resto sono entrambi nulli.
	\item Se $a$ ha un grado minore di $b$, il quoziente è nullo e il resto è $a$ stesso.
	\item Se $a$ ha un grado maggiore di $b$, allora abbiamo una divisione ``non banale'' e ci aspettiamo un quoziente che ha come grado la differenza tra quelli di $a$ e $b$.
\end{itemize}

Quando il resto della divisione è nullo, ossia $a=qb$, diremo che $b$ \emph{divide} $a$, e lo indicheremo con la notazione $a\dvd b$.
Poich\'e gli elementi dell'ideale principale $(a)$ sono della forma $ax$ per $x\in A$, vediamo subito che $a\dvd b$ implica $b\in(a)$, e viceversa.
\begin{definizione} \label{d:mcd}
	Dato un campo $K$, siano $a,b\in K[x]$ due polinomi non nulli.
	Si dice \emph{massimo comune divisore} tra $a$ e $b$ ogni polinomio $d\in K[x]$ tale che:
	\begin{itemize}
		\item $d$ divide sia $a$ che $b$;
		\item se un altro polinomio $c$ divide $a$ e $b$, allora divide anche $d$.
	\end{itemize}
\end{definizione}
Questa definizione ricalca quella del massimo comune divisore tra numeri interi.
Per i numeri in $\Z$, però, è noto che se $z$ è il massimo comune divisore tra $m$ e $n$, allora lo è anche $-z$.
Anche per i polinomi vale un risultato simile: dato un massimo comune divisore in $K[x]$, tutti i suoi multipli per una costante in $K$ lo sono ancora.
\begin{teorema} \label{t:altri-mcd}
	Siano $a,b\in K[x]$ non nulli, e $d$ un massimo comun divisore tra i due.
	Se $d'$ è un altro massimo comune divisore, allora vale la relazione $d'=kd$ per qualche $k\in K\setminus\{0\}$.
\end{teorema}
\begin{proof}
	Per la definizione di massimo comune divisore, $d$ e $d'$ dividono entrambi $a$ e $b$, e si dividono a vicenda, ossia $d\dvd d'$ e $d'\dvd d$.
	Dunque esistono $\alpha,\beta\in K[x]$ tali che $d=\alpha d'$ e $d'=\beta d$.
	Otteniamo da queste due uguaglianze che $d=\alpha\beta d$.
	Poich\'e $d\neq 0$ --- altrimenti dovrebbe essere $a=0$, contro le ipotesi fatte --- per le leggi di cancellazione si ha $\alpha\beta=1$.
	La somma dei gradi di $\alpha$ e $\beta$ deve quindi essere nulla, cioè il grado del polinomio unità: l'unico modo possibile affinché accada è che $\deg\alpha=\deg\beta=0$, ossia che entrambi siano, oltre che polinomi, anche elementi (scalari) del campo $K$, cioè $\alpha=k\in K$ e $\beta=h\in K$.
	Ma allora $hk=1$, cioè $h$ e $k$ sono invertibili, perciò non possono essere nulli.
	Dunque $d'=kd$ con $k\in K\setminus\{0\}$.
\end{proof}
A causa di questa arbitrarietà nella costante moltiplicativa, è conveniente stabilire la seguente definizione.
\begin{definizione}\label{d:mcd-monico}
	Dati $a,b\in K[x]$ non nulli, il massimo comune divisore di $a$ e $b$ con coefficiente direttivo unitario è detto \emph{massimo comune divisore monico}.
\end{definizione}
D'ora in poi, quando ci riferiremo al massimo comune divisore, sottintenderemo che prendiamo quello monico.

\begin{teorema}[Algoritmo di Euclide] \label{t:esistenza-mcd}
	Dato un campo $K$, e $a,b\in K[x]$ non nulli, esiste sempre un massimo comune divisore tra di loro.
\end{teorema}
\begin{proof}
	Dato che $b\neq 0$, esistono $q_1,r_1\in K[x]$ tali da poter scrivere $a=q_1b+r_1$ e per cui $\deg r_1<\deg b$.
	Se $r_1=0$, allora $a=qb$, quindi $b\dvd a$ e ovviamente anche $b\dvd b$; se esiste un $c\in K[x]$ tale che $c\dvd a$, esso è $b$ che è quindi il massimo comune divisore. % Da migliorare...
	
	Se $r_1\ne 0$, dividiamo $b$ per esso: esistono $q_2,r_2\in K[x]$ per cui $\deg r_2<\deg r_1$ e
	\begin{equation}
		b=q_2r_1+r_2.
	\end{equation}
	Se adesso $r_2=0$, troviamo che $r_1$ è il massimo comune divisore.
	Infatti, $r_1\dvd b$ dal fatto che $b=r_1q_2$, inoltre $a=q_1b+r_1=q_1(q_2r_1)+r_1=(q_1q_2+1)r_1$ dunque $r_1\dvd a$.
	Prendiamo dunque un $c\in K[x]$ che divida sia $a$ che $b$: ciò significa che esistono $\alpha,\beta\in K[x]$ tali che $a=c\alpha$ e $b=c\beta$.
	\begin{equation}
		c\alpha=a=q_1b+r_1=q_1c\beta+r_1\then r_1=(\alpha-q_1\beta)c
	\end{equation}
	ossia $c\dvd r_1$.
	Il polinomio $r_1$ soddisfa dunque la definizione \ref{d:mcd} di massimo comune divisore.

	Se invece $r_2\ne 0$, ancora possiamo dividere $r_1$ per $r_2$, dato che esistono $q_3,r_3\in K[x]$ tali che $\deg r_3<\deg r_2$ e
	\begin{equation}
		r_1=q_3r_2+r_3.
	\end{equation}
	Se $r_3=0$, allora esattamente come prima si dimostra che $r_2$ è il massimo comune divisore tra $a$ e $b$.
	Se $r_3\ne 0$, iteriamo ancora una volta dividendo $r_2$ per $r_3$ e distinguendo i casi se il resto di questa divisione è nullo o no.

	Il procedimento deve necessariamente avere un termine, in quanto a partire da $\deg b$ si ha la successione decrescente
	\begin{equation}
		\deg b>\deg r_1>\deg r_2>\dots
	\end{equation}
	e si giunge dopo un numero finito di passi con un resto nullo.
	Sia dunque $r_k=0$: abbiamo che
	\begin{equation}
		\begin{gathered}
			r_{k-3}=q_{k-1}r_{k-2}+r_{k-1}\\
			r_{k-2}=q_kr_{k-1}
		\end{gathered}
	\end{equation}
	perciò $r_{k-1}\dvd r_{k-2}$.
	Dalll'equazione precedente vediamo allora che $r_{k-1}\dvd r_{k-3}$ e cos\`i via per tutti i resti precedenti, fino a dividere anche $b$ e dunque $a$ dalla prima equazione $a=q_1b+r_1$.
	Infine, se un $c\in K[x]$ dividesse $a$ e $b$, allora dividerebbe $r_1$: ma allora divide anche $r_2$ (con un ragionamento analogo al precedente) e cos\`i via si vede che divide tutti i resti fino a $r_{k-1}$.
	Dunque il massimo comun divisore di $a$ e $b$ è proprio $r_{k-1}$.
\end{proof}

\begin{teorema}[Identità di B\'ezout] \label{t:bezout}
	Dato un campo $K$ e $a,b\in K[x]$, se $d$ è il loro massimo comune divisore, allora esistono $\xi,\eta\in K[x]$ tali per cui si ha
	\begin{equation}
		d=\xi a+\eta b.
		\label{eq:bezout}
	\end{equation}
\end{teorema}
\begin{proof}
	La determinazione di tali $\xi$ e $\eta$ si può fare tramite l'algoritmo di Euclide delle divisioni successive.
	Se il massimo comun divisore è uno tra $a$ o $b$, la tesi è ovvia, prendendo $\xi=1$ e $\eta=0$ o viceversa.
	
	Effettuiamo la prima divisione ottenendo $a=q_1b+r_1$, da cui $r_1=a-q_1b$.
	Se $r_1$ è il massimo comune divisore, ci basta porre $\xi=1$ e $\eta=-q_1$.
	Altrimenti, seguendo il teorema precedente dividiamo $b$ (sappiamo che $r_1\ne 0$, altrimenti $b$ sarebbe il massimo comun divisore) come $b=q_2r_1+r_2$.
	Se $r_2$ è il massimo comune divisore,
	\begin{equation}
		r_2=b-q_2r_1=b-q_2(a-q_1b)=-q_2a+(1-q_1q_2)b
	\end{equation}
	perciò poniamo $\xi=-q_2$ e $\eta=1-q_2q_2$ per trovare la \eqref{eq:bezout}.
	Altrimenti dividiamo ancora $r_1$ per $r_2$ (anche stavolta, $r_2\ne 0$ perch\'e $r_1$ non è il massimo comun divisore) e procediamo, una volta trovato il massimo comune divisore, ad esprimerlo tramite i resti delle divisioni precedenti fino a risalire ad $a$ e $b$.
	Anche in questo caso, come nell'algoritmo di Euclide, il numero di iterazioni è finito quindi in un numero finito di passi siamo sicuri di trovare due termini $\xi$ e $\eta$ che soddisfino la \eqref{eq:bezout}.
\end{proof}
Vediamo un esempio pratico: siano $a(x)=x^3-5$ e $b(x)=x^2+4$, due polinomi in $\Q[x]$.
Dividiamo $a$ per $b$ con l'algoritmo di Euclide, ottenendo
\begin{equation*}
	\begin{aligned}
		x^3-5&=x\cdot(x^2+4)+(4x-5)\\
		x^2+4&=\Big(\frac14x+\frac5{16}\Big)(4x-5)+\frac{41}{16}\\
		4x-5&=\Big(\frac{64}{41}x-\frac{80}{41}\Big)\cdot\frac{41}{16}
	\end{aligned}
\end{equation*}
perciò $\frac{41}{16}$, ossia $1$, (se lo prendiamo monico), è il massimo comune divisore di $a$ e $b$.
\begin{comment}
Risaliamo da esso ai polinomi di partenza sostituendo i vari resti dalle equazioni precedenti:
\begin{equation*}
	\begin{split}
		4x-5&=\Big(\frac{64}{41}x-\frac{80}{41}\Big)\cdot\frac{41}{16}=\\
		&=\Big(\frac{64}{41}x-\frac{80}{41}\Big)\Big[x^2+4-\Big(\frac14x+\frac5{16}\Big)(4x-5)\Big]=\\
		&=\Big(\frac{64}{41}x-\frac{80}{41}\Big)\Big\{x^2+4-\Big(\frac14x+\frac5{16}\Big)[x^3-5-x(x^2+4)]\Big\}
	\end{split}
\end{equation*}
da cui si ricava l'identità di B\'ezout per $a$ e $b$
\begin{equation*}
	\begin{split}
		4x-5&=-\Big(\frac{64}{41}x-\frac{80}{41}\Big)\Big(\frac14x+\frac5{16}\Big)(x^3-5)+\Big(\frac{64}{41}x-\frac{80}{41}\Big)\Big[1+x\Big(\frac14x+\frac5{16}\Big)\Big](x^2+4)=\\
		&=\Big(-\frac{16}{41}x^2-\frac{25}{41}\Big)(x^3-5)+\Big(\frac{16}{41}x^3+\frac{39}{41}x-\frac{80}{41}\Big)(x^2+4)		
	\end{split}
\end{equation*}
\end{comment}

\section{Polinomi primi e irriducibili}
\begin{definizione} \label{d:polinomio-primo}
	Dato $a\in K[x]$ con $\deg a>0$, esso si dice \emph{primo} se ogniqualvolta $a\dvd bc$ allora $a\dvd b$ o $a\dvd c$.
\end{definizione}
Il seguente lemma mostra un legame tra i polinomi primi e i corrispettivi ideali principali generati da essi.
\begin{lemma} \label{l:ideale-polinomio-primo}
	Dato $a\in K[x]$ con $\deg a>0$, l'ideale $(a)$ è primo se e solo se $a$ è primo.
\end{lemma}
\begin{proof}
	Siano $b,c\in K[x]$.
	Se $(a)$ è primo e $a\dvd bc$, allora $bc\in(a)$.
	Per la definizione di ideale primo, però, ciò implica che $b\in(a)$ o $c\in(a)$, ossia $a\dvd b$ o $a\dvd c$.
	Dunque $a$ è primo.

	Sia ora $a$ un polinomio primo: se $a\dvd bc$ allora $a\dvd b$ oppure $a\dvd c$.
	Ma $a\dvd bc$ implica $bc\in(a)$, e analogamente $a\dvd x$ implica $x\in(a)$.
	Allora $(a)$ è un ideale primo.
\end{proof}

\begin{definizione} \label{d:polinomio-irriducibile}
	Sia $a\in K[x]$ con $\deg a=n>0$.
	Esso si dice \emph{irriducibile} se è divisibile solo per i polinomi $c\in K[x]$ con $\deg c=0$ e per quelli della forma $\lambda a$, con $\lambda\in K\setminus\{0\}$.
\end{definizione}
Notiamo subito che tutti i polinomi di grado 1, ossia della forma $p(x)=x-\alpha$, sono sempre irriducibili.

\begin{teorema} \label{t:polinomio-irriducibile-primo}
	Dato un campo $K$, un polinomio in $K[x]$ di grado positivo è irriducibile se e solo se è primo.
\end{teorema}
\begin{proof}
	Sia $a\in K[x]$ irriducibile: prendiamo $b,c\in K[x]$ e supponiamo che $a\dvd bc$.
	Mostriamo che se $a$ non divide $b$, allora $a\dvd c$.
	Escludiamo il caso $b=0$, per il quale si avrebbe che $a\dvd b$: abbiamo quindi $a,b\ne 0$.
	Sia dunque $d$ il massimo comune divisore tra $a$ e $b$.
	Essendo $a$ irriducibile, abbiamo che o $d=\lambda$ o $d=\mu a$, per $\lambda,\mu\in K\setminus\{0\}$.
	Il secondo caso non è possibile, perch\'e a quel punto $a=\mu^{-1}d$ quindi dividerebbe $b$.
	Dunque $d=\lambda$, con $\lambda=1$ prendendo il polinomio monico.
	Per l'identità di B\'ezout \ref{t:bezout}, abbiamo allora
	\begin{equation*}
		d=1=\xi a+\eta b
	\end{equation*}
	per qualche $\xi,\eta\in K[x]$.
	Moltiplicando per $c$, otteniamo $c=c\xi a+c\eta b$: poich\'e per ipotesi però $a\dvd bc$, esiste $g\in K[x]$ tale per cui $ga=bc$.
	Allora
	\begin{equation*}
		c=c\xi a+\eta ga=(c\xi+\eta g)a
	\end{equation*}
	ossia $a\dvd c$.

	Sia ora $a\in K[x]$ primo: se fosse riducibile, allora $\exists f,g\in K[x]$ (diversi da multipli scalari di $a$) con $\deg f,\deg g>0$ tali che $a=fg$: allora $a\dvd fg$.
	Essendo primo, però, divide sicuramente uno dei due, sia esso $f$: esiste quindi $\xi\in K[x]$, non nullo, per il quale $f=a\xi$.
	Ma allora
	\begin{equation*}
		a=fg=a\xi g \then (1-\xi g)a=0
	\end{equation*}
	ossia $\xi g=1$: di conseguenza $\deg\xi+\deg g=0$.
	Dato che $\deg g>0$ e $\deg\xi\ge 0$, questo è assurdo: ciò prova che non esistono $f,g$ diversi da multipli scalari di $a$ e di grado positivo che dividono $a$, che quindi non è riducibile.
\end{proof}

\begin{teorema}[di fattorizzazione unica] \label{t:fattorizzazione-unica}
	Ogni polinomio $a\in K[x]$, con $\deg a>0$, può essere scritto come un prodotto
	\begin{equation}
		up_1\cdots p_n
	\end{equation}
	dove $u\in K$ e tutti i $p_i$ sono irridubicili.
	La fattorizzazione è essenzialmente unica, nel senso che se $a=kp_1\cdots p_n=hq_1\cdots q_t$ con $h,k\in K$ e $p_i,q_i$ irriducibili, allora $n=t$ e a meno di permutazioni $p_i=cq_i$ con $c\in K$ per ogni $i$.
\end{teorema}
\begin{proof}
	Dimostriamo per induzione su $n\defeq\deg a$.

	Per prima cosa sia $n=1$: per quanto già detto, essendo di primo grado è già irriducibile, quindi è la fattorizzazione cercata, che è ovviamente anche unica.

	Supponiamo che la tesi sia vera da 1 a $n$.
	Se $a$ è irriducibile, la dimostrazione è conclusa, altrimenti scriviamo $a=gh$ per qualche $g,h\in K[x]$.
	Se $g$ e $h$ sono entrambi irriducibili abbiamo trovato la fattorizzazione; se non è questo il caso, almeno uno tra $g$ e $h$ ha comunque un grado minore di quello di $a$, e per l'ipotesi di induzione è dunque riducibile.
	Procediamo scomponendo $g$ o $h$, fino a trovare $a=p_1p_2\dots p_n$, dove ogni $p_i$ è irriducibile.
	
	Ammettiamo che esistano due fattorizzazioni
	\begin{equation}
		a=up_1p_2\cdots p_n=vq_1q_2\cdots q_t.
	\end{equation}
	con $u,v\in K$.
	Poich\'e $p_1$ divide $a$, divide uno dei fattori $q_i$ al secondo membro, e sicuramente non $v$.
	Riordiniamoli in modo che $p_1\dvd q_1$: poich\'e $q_1$ è irriducibile, ciò significa che $q_1=k_1p_1$, con $k_1\in K$ (l'ipotesi $p_1\in K$ è chiaramente da escludere).
	Troviamo allora 
	\begin{equation}
		up_1\cdots p_n=vk_1p_1q_2\cdots q_t.
	\end{equation}
	ed essendo $K[x]$ un dominio d'integrità e $p_1\ne 0$ possiamo dividere per esso ottenendo
	\begin{equation}
		up_2\cdots p_n=vk_1q_2\cdots q_t.
	\end{equation}
	Ora al primo membro abbiamo $n-1<n$ fattori $p_i$, e proseguendo con lo stesso ragionamento precedente vediamo che al secondo ne abbiamo $t-1=n-1$, dunque per l'ipotesi di induzione abbiamo che $p_i=k_iq_i$ per qualche $k_i\in K$ per ogni $2\le i\le n$, e ciò prova la tesi.
\end{proof}
A questo punto, possiamo dividere tutti i fattori irriducibili per delle costanti opportune in $K$ per renderli monici, come nel seguente corollario.
\begin{corollario} \label{c:fattorizzazione-polinomi-monici}
	Dato $a\in K[x]$, con $\deg a=s>0$, esso si può sempre scrivere univocamente come $a=ka_1\dots a_s$, in cui
	\begin{itemize}
		\item $k\in K\setminus\{0\}$ è il coefficiente direttivo di $a$;
		\item $a_i$, $\forall i\in\{1,\dots,s\}$ è monico e irriducibile.
	\end{itemize}
\end{corollario}
\begin{proof}
	Preso $a\in K[x]$, possiamo esprimerlo come $a=k_1a_1$ con $a_1$ monico.
	Se si esegue lo stesso ragionamento su $a=p_1\dots p_t$, nel teorema precedente, si ottiene $a=k_1k_2\dots k_sa_1\dots a_s$ con i vari $a_i$ monici e irriducibili $\forall i\in\{1,\dots,n\}$, si ha che $k_1k_2\dots k_s\in K$, perciò deve essere il coefficiente direttivo del polinomio corrispondente.
\end{proof}   

\section[Domini a ideali principali]{Domini a ideali principali\footnote{
	Questa sezione è diversa da come è stata affrontata a lezione: cambia il modo in cui si arriva al teorema \ref{t:ideale-massimale-sse-irriducibile}.
	In questa versione è aggiunto il teorema \ref{t:pid-ideale-massimale-sse-primo}, non visto a lezione, e grazie ad esso la dimostrazione del \ref{t:ideale-massimale-sse-irriducibile} diventa facilissima.
	Questo e altro è tratto da \cite{PID}.
	I due teoremi citati sono dunque differenti, mentre il resto dei teoremi e corollari è rimasto sostanzialmente invariato.
}}
\begin{definizione} \label{d:dominio-ideali-principali}
	Un dominio d'integrità $A$ è detto \emph{a ideali principali} se è tutti i suoi ideali propri sono principali, ossia se per ogni ideale $I\subset A$ esiste $a\in A$ per cui $I=(a)$.
\end{definizione}
Dimostriamo subito l'inverso del corollario \ref{c:ideale-massimale-primo} che avevamo anticipato.
\begin{teorema} \label{t:pid-ideale-massimale-sse-primo}
	In un dominio a ideali principali, un ideale è primo se e solo se è massimale.
\end{teorema}
\begin{proof}
	Abbiamo già dimostrato nel corollario \ref{c:ideale-massimale-primo} che se un ideale è primo, allora è anche massimale.
	Sia $A$ un dominio a ideali principali, $I$ un suo ideale primo e $J$ un altro ideale tali che $I\subseteq J\subseteq A$.
	Sappiamo che esistono $a,b\in A$ tali che $I=(a)$ e $J=(b)$.
	Poich\'e $(a)\subseteq(b)$, si ha $a\in(b)$ quindi esiste $x\in A$ tale che $a=xb$: allora $a\dvd xb$.
	L'ideale $I$ è primo, quindi anche $a$ è un polinomio primo, perciò abbiamo che $a\dvd b$ oppure $a\dvd x$.
	Nel primo caso, si ha $b\in(a)$ perciò $(a)=(b)$, ossia $I=J$.
	Nel secondo caso, esiste $y\in A$ tale che $ya=x$, ma allora $x=y(xb)=x(yb)$.
	Poich\'e $A$ è un dominio di integrità, e $x\ne 0$, ciò implica che $yb=1$, e di conseguenza $1\in(b)$.
	Ma un ideale che contiene l'unità coincide con l'anello intero, perciò $(b)=A$.
	Ciò prova che $I$ è massimale.
\end{proof}
Mostriamo ora che possiamo sfruttare molte utili proprietà, come la completa equivalenza tra ``primo'' e ``irriducibile'', nello studio degli anelli di polinomi in una incognita, in virtù del seguente teorema.
\begin{teorema} \label{t:pid-anello-polinomi-su-campo}
	Dato un campo $K$, l'anello $K[x]$ è un dominio a ideali principali.
\end{teorema}
\begin{proof}
	Sia $I$ un ideale di $K[x]$.
	Se $I=\{0\}$, ovviamente $I=(0)$ quindi è un ideale principale.
	Se $I\ne\{0\}$ allora in esso ci sono dei polinomi di grado maggiore di zero.
	Poniamo
	\begin{equation*}
		m\defeq\min\{\deg p\colon p\in I\}
	\end{equation*}
	che esiste per il principio del buon ordinamento.\footnote{Il principio del buon ordinamento afferma che \emph{ogni insieme di numeri naturali non vuoto contiene un numero che è più piccolo di tutti gli altri}.}
	Sia $g\in I\setminus\{0\}$ tale che $\deg g=m$: vogliamo mostrare che $(g)=I$.
	Chiaramente $(d)\subseteq I$ dalla definizione di ideale.
	Prendiamo inolte $y\in I$: certamente esistono $q,r\in K[x]$ tali per cui $\deg r<\deg y$ e $y=qg+r$.
	Di conseguenza, $r=y-qg\in I$: se però $r\ne 0$, avremmo $\deg r<\deg g$ che viola l'ipotesi fatta ($g$ ha il grado minore tra tutti i polinomi di $I$).
	Deve necessariamente essere allora $r=0$, da cui $y=qg$.
	Ma allora $g\dvd y$, e poich\'e questo vale per ogni $y\in I$ risulta $I\subseteq (d)$.
	Ciò prova che $I=(d)$, perciò ogni ideale di $K[x]$ è un ideale principale.
\end{proof}
\begin{corollario} \label{c:unicita-generatore-monico}
	Ogni ideale principale $I\in K[x]$, con $I\ne (0)$, ha un unico generatore monico.
\end{corollario}
\begin{proof}
	Poniamo $I=(g)$ con $g(x) = a_nx^n+\dots+a_0$ con $a_n\ne 0$.
	Possiamo allora moltiplicare $g$ per $a_n^{-1}$, ottenendo che $I$ è anche generato da $\tilde{g} = a_n^{-1}g$ che è monico, cioè $(\tilde{g}) = (g)$: ciò prova l'esistenza di un generatore monico di $I$.
	Dimostriamone l'unicità: sia $I=(h)$, con $h$ monico.
	Poich\'e $(h)=(\tilde{g})$, risulta che $\tilde{g}\dvd h$ e $h\dvd \tilde{g}$ ossia esistono $\alpha,\beta\in K[x]$ per cui $h=\alpha\tilde{g}$ e $\tilde{g}=\beta h(x)$.
	Quindi $h=\alpha\beta h$, e poich\'e $I\ne\{0\}$ e $K[x]$ è un dominio d'integrità risulta $\alpha\beta=1$.
	Dunque $\alpha,\beta\in K\setminus\{0\}$: $h=\alpha\tilde{g}$ per ipotesi è monico, e dato che lo è anche $\tilde{g}$ si ottiene $\alpha=1$.
	Di conseguenza $h=\tilde{g}$, cioè il generatore monico è unico.
\end{proof}
\begin{teorema} \label{t:rappresentazione-laterali-ideale-principale}
	Sia $(g)$ un ideale non nullo di $K[x]$.
	Ogni classe laterale di $(g)\in K[x]\quot g$ si può rappresentare univocamente nella forma $(g)+r$, con $\deg r<\deg g$.
\end{teorema}
\begin{proof}
	Una classe laterale dell'ideale $(g)$ è del tipo $(g)+f$, per $f\in K[x]$.
	Per il teorema \ref{t:esistenza-mcd} esistono sempre $q,r\in K[x]$, con $\deg r<\deg g$ tali che $f=qg+r$.
	Si ha allora
	\begin{equation*}
		(g) + f = (g) + qg + r = (g) + r,
	\end{equation*}
	dato che $qg\in (g)$, quindi un tale $r$ esiste.
	Mostriamo che è unico.
	Supponiamo che esista anche un $r'\in K[x]$ tale che la classe laterale si possa rappresentare come $(g)+r'$, con $\deg r'<\deg g$.
	Dalle proprietà \ref{pr:gradi-polinomi-operazioni} risulta
	\begin{equation*}
		\deg (r'-r)\leq\max\{\deg r',\deg r\}<\deg g.
	\end{equation*}
	Ora, nel caso $\deg(r'-r)\ge 0$, se fosse $r'-r\in(g)$ allora si avrebbe $r'-r=hg$ per qualche $h\in K[x]$, ma allora $\deg(r'-r)=\deg(hg)$ e contemporaneamente
	\begin{equation*}
		\deg(r'-r)<\deg g\ge \deg(hg)
	\end{equation*}
	che porta ad una contraddizione.
	Dunque $\deg(r'-r)=0$, ossia $r'-r=0=0\cdot g$ perciò $r'-r\in(g)$.
	Avendo i due rappresentanti in relazione, le due classi laterali sono equivalenti.
\end{proof}
\begin{corollario}
	Sia $K$ un campo finito e $g\in K[x]$ di grado $n>0$.
	Allora $|K[x]\quot(g)|=|K|^n$.
\end{corollario}
\begin{proof}
	Sapendo che le classi laterali sono scritte come $(g)+r(x)$, ora ipotizziamo due casi, cioè $\deg r(x) = 0$, oppure $\deg r(x) = -1$, ricaviamo:
	\begin{align*}
		\deg r(x) &= 0 \text{ si ha $(g) + r(x) = (g) + a_0$,}\\
		\deg r(x) &= 1 \text{ si ha $(g) + r(x) = (g) + a_1x + a_0$.}
	\end{align*}
	Nel primo caso si ritrovano tutte le possibili combinazioni degli $a_0\in K$, che sono $m$, nel secondo caso le possibili combinazioni sono $m^2$. Si può arrivare dunque a dimostrare la tesi.
\end{proof}
\begin{teorema} \label{t:ideale-massimale-sse-irriducibile}
	Sia $g\in K[x]$ con $\deg g>0$.
	L'ideale $(g)$ è massimale se e solo se $g$ è irriducibile.
\end{teorema}
\begin{proof}
	Non bisogna far altro che applicare dei teoremi già visti in precedenza:
	\begin{equation*}
		(g)\text{ è massimale }\iff (g)\text{ è primo }\iff g\text{ è primo }\iff g\text{ è irriducibile}
	\end{equation*}
	per i teoremi, in ordine, \ref{t:pid-ideale-massimale-sse-primo}, \ref{l:ideale-polinomio-primo} e \ref{t:polinomio-irriducibile-primo}.
\end{proof}
\begin{corollario}
	Dato $g\in K[x]$ con $\deg g>0$, $K[x]\quot(g)$ è un campo se e solo se $g$ è irriducibile.
\end{corollario}
\begin{proof}
	Dal teorema precedente abbiamo che $g$ è irriducibile se e solo se $(g)$ è massimale.
	Collegando anche il teorema \ref{t:ideale-massimale-quoziente-campo} troviamo la tesi.
\end{proof}
Vediamo un esempio concreto delle conseguenze di questi teoremi.
Partiamo dall'anello $\R[x]$ dei polinomi a coefficienti reali, e un polinomio irriducibile di grado maggiore di 1, come $x^2+1$.
Essendo irriducibile, l'ideale $(x^2+1)$ è primo e massimale, e $\R[x]\quot (x^2+1)$ un campo.
I suoi elementi, dal teorema \ref{t:rappresentazione-laterali-ideale-principale}, si scrivono tutti come un elemento di $(x^2+1)$ più un polinomio di primo grado, ossia se $p\in\R[x]\quot(x^2+1)$ allora
\begin{equation}
	p=(x^2+1)+ax+b.
\end{equation}
Prendiamo un'altro elemento $q=(x^2+1)+cx+d$.
La somma di due elementi è definita in modo naturale come
\begin{equation}
	p+q=(x^2+1)+ax+b+(x^2+1)+cx+d=(x^2+1)+(a+c)x+b+d
\end{equation}
e il prodotto come
\begin{equation}
	pq=[(x^2+1)+ax+b]\cdot[(x^2+1)+cx+d]=(x^2+1)+acx^2+(ad+bc)x+bd.
\end{equation}
Il termine $acx^2$ però ha lo stesso grado di $x^2+1$, quindi non deve comparire.
Possiamo in effetti trovare un modo per eliminarlo: aggiungendo e sottraendo $ac$ al risultato, otteniamo
\begin{equation}
	\begin{split}
		pq&=(x^2+1)+acx^2+ac+(ad+bc)x+bd-ac=\\
		&=(x^2+1)+ac(x^2+1)+(ad+bc)x+bd-ac=\\
		&=(x^2+1)+(ad+bc)x+bd-ac
	\end{split}
\end{equation}
dato che $ac(x^2+1)\in(x^2+1)$ quindi viene ``assorbito'' dall'ideale.

Prendiamo ora l'insieme $\R^2$ delle coppie di numeri reali, e dotiamolo delle operazioni
\begin{equation}
	\begin{gathered}
		(b,a)+(\beta,\alpha)=(b+\beta,a+\alpha)\\
		(b,a)(\beta,\alpha)=(a\beta+b\alpha, b\beta-a\alpha).
	\end{gathered}
\end{equation}
Questa struttura individua un ulteriore campo, di cui non è difficile notare il legame con il precedente $\R[x]\quot (x^2+1)$.
Troviamo infatti un isomorfismo $\gamma\colon\R[x]\quot(x^2+1)\to\R^2$, definito come
\begin{equation}
	\gamma\colon (x^2+1)+ax+b\mapsto (b,a)
\end{equation}
che li lega.
Ovviamente quest'ultimo campo $\R^2$, con le operazioni definite, non è altro che il campo complesso come costruito da Hamilton, ma con la coppia $(a,b)$ in ordine contrario.
Per passare alla notazione comune $a+ib$ non dobbiamo far altro che definire un nuovo insieme $\hat{\C}=\{a+ib\colon a,b\in\R\}$ con le note operazioni, e tale che $i^2=-1$.
L'isomorfismo che mette in relazione i due campi è evidentemente un $\phi\colon\C\to\hat{\C}$ per il quale
\begin{equation}
	\phi(b,a)=a+ib.
\end{equation}
\section{Radici di un polinomio}
\begin{definizione} \label{d:radice-polinomio}
	Dato un anello $A$ commutativo e con unità e un polinomio $p\in A[x]$, si dice \emph{radice} di $p$ un elemento $\alpha\in A$ per cui $p(\alpha)=0$.
\end{definizione}

\begin{teorema}[di Ruffini] \label{t:ruffini}
	Dato un campo $K$ e un polinomio $p\in K[x]$, $\alpha$ è una radice di $p$ se e solo se $(x-\alpha)\dvd p$.
\end{teorema}
\begin{proof}
	Sia $\alpha$ una radice di $p$.
	Possiamo dividere $p$ per $x-\alpha$, il cui grado non è nullo, ottenendo che
	\begin{equation*}
		p(x)=q(x)(x-\alpha)+r(x)
	\end{equation*}
	con $\deg r<\deg\big( (x-\alpha)\big)=1$.
	Dato che $\deg r\in\{0,1\}$, quindi, $r(x)=k$ per qualche $k\in K$, eventualmente $k=0$ se $\deg r=-1$.
	Ma essendo $\alpha$ una radice di $p$, valutando $p(\alpha)$ otteniamo
	\begin{equation*}
		0=p(\alpha)=q(\alpha)(\alpha-\alpha)+k=k
	\end{equation*}
	perciò $k=r(x)=0$: di conseguenza $p(x)=q(x)(x-\alpha)$, ossia $(x-\alpha)\dvd p$.
	
	Sia ora $(x-\alpha)\dvd p$: possiamo dunque scrivere $p(x)=g(x)(x-\alpha)$ per qualche $g\in K[x]$.
	Ma allora, valutandolo in $\alpha$, risulta
	\begin{equation*}
		p(\alpha)=g(\alpha)(\alpha-\alpha)=0
	\end{equation*}
	quindi $\alpha$ è una radice di $p$.
\end{proof}
Per esempio, sia $f(x) = a_1 x + a_0 \in K[x]$ con $a_1 \neq 0$.
Si ha che $\alpha = - \frac{a_0}{a_1}$ è sempre una radice.

Si può, visti i teoremi precedenti, porre una relazione tra la presenza di una radice e la possibilità di ridurre un polinomio. Sia $f(x)\in K[x]$ e $\deg f(x) > 1$, se $f(x)$ ammette una radice $\alpha$, allora $f(x)$ deve essere riducibile come $f(x) = (x-\alpha) g(x)$.

Non è detto, in generale, che ogni polinomio riducibile abbia necessariamente una radice: basta prendere in $\R[x]$ il polinomio $x^4+2x^2+1$.
Esso si può scomporre in $(x^2+1)(x^2+1)$, che chiaramente non hanno radici reali.
Se invece il polinomio \emph{è riducibile} e ha grado 2 o 3, allora certamente ha una radice: in fatti almeno uno dei fattori in cui è scomposto deve avere grado 1, cioè sarà della forma $x-\lambda$, perciò tale $\lambda$ è una radice.

Un caso importante è quello dei numeri complessi: in tale campo, si può sempre scomporre un polinomio (non costante) in un prodotto di opportuni polinomi di primo grado, per via del seguente teorema (che non dimostriamo).
\begin{teorema}[Teorema fondamentale dell'algebra] \label{t:fondamentale-algebra}
	Ogni polinomio in $\C[x]$ di grado positivo ammette sempre una radice in $\C$.
\end{teorema}

\begin{definizione} \label{d:molteplicita-algebrica-radice}
	Siano $K$ un campo, $f\in K[x]$ e $\alpha\in K$.
	Si dice che $\alpha$ è una radice di $f$ con \emph{molteplicità algebrica} $r\in \N$ se $(x-\alpha)^r\dvd f$ ma $(x-\alpha)^{r+1}$ non divide $f$.
\end{definizione}
In particolare, una radice di molteplicità algebrica 1 è detta \emph{semplice}.
\begin{teorema} \label{t:somma-molteplicita-algebriche}
	Sia $f\in K[x]$ con $\deg f\ge 0$.
	Date le radici distinte $\alpha_1,\dots,\alpha_k$ di $f$ con molteplicità algebrica rispettivamente $r_1,\dots,r_k$, si ha che $\sum_{i=1}^n r_i\le\deg f$.
\end{teorema}
\begin{proof}\footnote{
		Questa dimostrazione differisce, anche se non troppo, da quella svolta a lezione.
	}
	Secondo il teorema \ref{t:fattorizzazione-unica}, scriviamo $f$ come
	\begin{equation}
		f = p_1p_2\dots p_k,
	\end{equation}
	con ogni $p_i$ primo.
	Data una radice $\alpha_1$, si ha che $(x-\alpha_1)\dvd f$, quindi divide uno dei $p_i$.
	Riordiniamo l'ordine del prodotto in modo che $(x-\alpha_1)\dvd p_1$: poich\'e $p_1$ è primo, quindi irriducibile, dovrà essere della forma $h(x-\alpha_1)$ con $h\in K\setminus\{0\}$.
	Raccogliendo tutti i fattori di questo tipo nel prodotto otteniamo
	\begin{equation}
		f(x)=(x-\alpha_1)^{k_1}u(x)
	\end{equation}
	con ovviamente $k_1\ge r_1$ (altrimenti $r_1$ non sarebbe la molteplicità di $\alpha_1$), e $x-\alpha_1$ che non divide $u$.
	Allo stesso modo, però, $(x-\alpha_1)^{r_1}\dvd f$, dunque
	\begin{equation}
		f(x)=(x-\alpha_1)^{r_1}v(x).
	\end{equation}
	Eguagliando le due espressioni trovate abbiamo
	\begin{equation}
		(x-\alpha_1)^{k_1}u(x)=(x-\alpha_1)^{r_1}v(x)\quad\then\quad (x-\alpha_1)^{r_1-k_1}v(x)=u(x)
	\end{equation}
	dato che $K[x]$ è un dominio d'integrità.
	Se ora $r_1>k_1$, si avrebbe che $x-\alpha_1\dvd u$, ma ciò contrasta la scelta di $k_1$: allora $r_1=k_1$ da cui
	\begin{equation}
		f(x)=(x-\alpha_1)^{r_1}u(x).
	\end{equation}
	Passiamo alla radice $\alpha_2$: poich\'e $\alpha_2\ne\alpha_1$, certamente $x-\alpha_2$ non divide $(x-\alpha_1)^{r_1}$, quindi dovrà dividere $u$.
	Procediamo in questo modo fino ad esaurire le radici $\alpha_i$, giungendo a una forma
	\begin{equation}
		f(x)=(x-\alpha_1)^{r_1}\cdots(x-\alpha_k)^{r_k}g(x).
	\end{equation}
	Allora dalle proprietà \ref{pr:gradi-polinomi-operazioni} otteniamo
	\begin{equation}
		\deg f=\sum_{i=1}^kr_i+\deg g\ge\sum_{i=1}^kr_i
	\end{equation}
	come volevamo dimostrare.
\end{proof}

\begin{corollario}[Principio d'identità dei polinomi] \label{c:principio-identita-polinomi}
	Siano $\alpha_1, \dots, \alpha_{n+1}$ elementi distinti di $K$.
	Se $f,g\in K[x]$, al più di grado $n$, sono tali che $f(\alpha_i)=g(\alpha_i)$ $\forall i\in\{1,\dots,n+1\}$, allora $f=g$.
\end{corollario}
\begin{proof}
	Supponiamo per assurdo che sia $f\ne g$.
	Allora si ha che $f-g\ne 0$, perciò $n\ge\deg (f-g)\ge 0$.
	Se $f(\alpha_i)=g(\alpha_i)$ $\forall i\in\{1,\dots,n+1\}$, allora ogni $\alpha_i$ è radice di $f-g$, che ha quindi $n+1$ radici.
	Per il teorema precedente, però, risulterebbe $\deg(f-g)\ge\sum_{k=1}^{n+1}r_i\ge n+1$ (nel migliore dei casi, $\deg(f-g)=n+1$ se ogni radice è semplice), che è assurdo perch\'e come visto si ha $\deg(f-g)\le n$.
	Dunque deve essere $f-g=0$, ossia $f=g$.
\end{proof}

\chapter{Spazi vettoriali} \label{ch:spazi-vettoriali}
\section{Proprietà principali} \label{sec:proprieta-spazi-vettoriali}
\begin{definizione} \label{d:spazio-vettoriale}
	Dato un campo $K$, un insieme $V$ non vuoto e due operazioni interne $+\colon V\times V\to V$ e $\cdot\colon K\times V\to V$, la terna $(V,+,\cdot)$ si definisce \emph{spazio vettoriale} sul campo $K$ se sono soddisfatte le seguenti proprietà:
	\begin{itemize}
		\item $(V,+)$ è un gruppo abeliano;
		\item $1_K x=x$ per ogni $x\in V$;
		\item la proprietà associativa, ossia se $\forall\lambda,\mu\in K$ e $\forall x\in V$, si ha $\lambda(\mu x)=(\lambda\mu) x$;
		\item la proprietà distributiva, ossia se $\forall\lambda,\mu\in K$ e $\forall x,y\in V$, si ha $(\lambda+\mu)x=\lambda  x+\mu x$ e $\lambda(x+y)=\lambda  x+\lambda  y$.
	\end{itemize}
\end{definizione}
Gli elementi di $V$ si chiamano \emph{vettori} mentre quelli di $K$ \emph{scalari}.
L'elemento neutro della somma, che per le proprietà note dei gruppi esiste ed è unico, sarà indicato con $0$, oppure $0_V$ in caso di ambiguità.
Lo zero e l'unità del campo $K$ seguono la convenzione già usata per la quale saranno indicati con $0$ e $1$, o anche $0_K$ e $1_K$; il fatto che $0$ indichi sia lo zero di $K$ che quello di $V$ sarà spesso chiaro dal contesto.
\paragraph{Esempi}
\begin{itemize}
	\item $(\R^n,+,\cdot)$, l'insieme delle $n$-uple ordinate di numeri reali, è uno spazio vettoriale su $\R$, infatti $\forall\lambda\in\R$ si ha, rappresentando i vettori come colonne,
		\begin{equation*}
			\lambda\cdot
			\begin{pmatrix}
				x_1\\\vdots\\x_n
			\end{pmatrix}
			=
			\begin{pmatrix}
				\lambda x_1\\\vdots\\\lambda x_n
			\end{pmatrix}
		\end{equation*}
		eccetera definendo somma e prodotto per scalare componente per componente.
	\item L'anello dei polinomi $\R[x]$ è uno spazio vettoriale su $\R$ con l'addizione e il prodotto per un numero reale, dove moltiplicare un polinomio per $\lambda\in\R$ equivale a moltiplicare per tale scalare tutti i suoi termini.
	\item L'insieme delle funzioni (qualunque) definite da un insieme $X\ne\emptyset$ e a valori reali forma uno spazio vettoriale, con le operazioni di addizione e prodotto per numero reale ``puntuali'', ossia $(f+g)(x)=f(x)+g(x)$ e $(\lambda f)(x)=\lambda f(x)$.
	\item Ogni campo può essere visto come spazio vettoriale su se stesso: ad esempio $(\R,+,\cdot)$ è uno spazio vettoriale su $\R$, e $(\C,+,\cdot)$ è uno spazio vettoriale su $\C$ ma anche su $\R$ se lo consideriamo come l'insieme delle coppie $(a,b)=a+ib$ con $a,b\in\R$.
\end{itemize}

Elenchiamo ora una serie di proprietà di base sugli spazi vettoriali, in cui assumiamo $V$ come spazio vettoriale su un campo $K$.
\begin{proprieta} \label{p:annullamento-vettore}
	Per ogni vettore $x\in V$, $0_K x=0_V$.
\end{proprieta}
\begin{proof}
	Lo $0_K$ si può sempre scrivere come somma di $0_K$ con se stesso, quindi $0_K x=(0_K+0_K)x=0_K x+0_K x$.
	Poiché $V$ è abeliano, sommando l'inverso di $0_K  x$ ai due membri si ottiene $0_K x=0_V$
\end{proof}
\begin{proprieta} \label{p:scalare-opposto}
	Per ogni scalare $a\in K$ e $\forall x\in V$, $-(ax)=(-a)x$.
\end{proprieta}
\begin{proof}
	Per la proprietà precedente si ha $0_V=0_K x$, e lo zero scalare si scrive come somma degli inversi $a+(-a)$, quindi $0_V=[a+(-a)]x=ax+(-a)x$, che significa che $(-a)x$ è il vettore inverso di $ax$ rispetto alla somma, ossia $(-a)x=-(ax)$.
\end{proof}
\begin{proprieta} \label{p:scalare-per-vettore-nullo}
	Per ogni $a\in K$, $a0_V=0_V$.
\end{proprieta}
\begin{proof}
	Si ha che $a0_V=a(0_V+0_V)=a0_V+a0_V$, e come per la proprietà \ref{p:annullamento-vettore} poiché $V$ è abeliano si somma ai due membri dell'uguaglianza l'inverso di $a0_V$, ottenendo $a0_V=0_V$.
\end{proof}
\begin{proprieta} \label{p:annullamento-prodotto-scalare-vettore}
	Se $ax=0_V$ per $a\in K$ e $x\in V$, allora $a=0_K$ o $x=0_V$.
\end{proprieta}
\begin{proof}
	Se $a=0_K$ è ovvia, se invece $a\neq 0_K$ allora esiste il suo inverso, $a^{-1}\in K$, rispetto al prodotto in $K$ (cioè tale che $aa^{-1}=1_K$).
	Quindi $0_V=a^{-1}0_V$, e poiché per per ipotesi $ax=0$ segue che $0_V=a^{-1}(ax)=(aa^{-1})x=1_K x=x$, perciò $x=0_V$.
\end{proof}
\begin{proprieta} \label{p:cancellazione-vettore}
	Per ogni $a,b\in K$ e per ogni $x\in V$, se $ax=bx$ allora $a=b$ oppure $x=0_V$.
\end{proprieta}
\begin{proof}
	Se vale che $ax=bx$, allora aggiungendo l'inverso di $bx$ per la somma si ottiene $ax-(bx)=0_V$.
	Inoltre per la proprietà distributiva questo è uguale ad $ax+(-b)x=(a-b)x=0_V$.
	Per la proprietà \ref{p:annullamento-prodotto-scalare-vettore}, infine, $a+(-b)=0_K$ oppure $x=0_V$.
	Sommando $b$ alla prima delle due risulta $a=b$ o $x=0_V$.
\end{proof}
\begin{proprieta} \label{p:cancellazione-scalare}
	Per ogni scalare $\lambda\in K$ e $\forall x,y\in V$, se $\lambda x=\lambda y$ allora $\lambda=0_K$ o $x=y$.
\end{proprieta}
\begin{proof}
	Da $\lambda x=\lambda y$ risulta $\lambda x+\big(-(\lambda y)\big)=\lambda x+\lambda(-y)=0_V$.
	Per la proprietà distributiva equivale a $\lambda\big(x+(-y)\big)=0_V$, da cui sempre per la \ref{p:annullamento-prodotto-scalare-vettore} $\lambda=0_K$ oppure $  x+(-y)=0_V$, da cui sommando $y$ ai due membri risulta $\lambda=0_K$ oppure $x=y$.
\end{proof}

\section{Sottospazi vettoriali} \label{ref:sottospazi-vettoriali}
\begin{definizione} \label{d:sottospazio-vettoriale}
	Sia $V$ uno spazio vettoriale sul campo $K$.
	Un suo sottoinsieme $W\subseteq V$ non vuoto si dice \emph{sottospazio} vettoriale se $(W,+,\cdot)$, con le operazioni indotte da $V$, è a sua volta uno spazio vettoriale.
\end{definizione}
In altre parole un sottospazio (ometteremo spesso l'attributo ``vettoriale'' per brevità) è un sottoinsieme che risulta chiuso rispetto alle due operazioni dello spazio vettoriale che lo contiene.
Per verificare che un insieme $W$ sia un sottospazio bisogna dunque provare che le combinazioni lineari di elementi di $W$ siano ancora in $W$: una condizione necessaria facile da verificare è che $W$ deve contenere lo $0_V$.

Ogni spazio vettoriale $V$ contiene sempre due spazi vettoriali, che sono banalmente $\{0_V\}$ e $V$ stesso.
Vediamone altri esempi.
\begin{itemize}
	\item Preso lo spazio vettoriale $\R^n$, l'insieme
		\begin{equation*}
			N=\left\{
			\begin{pmatrix}
				x_1\\\vdots\\x_{n-1}\\0
			\end{pmatrix}
			\colon x_1,\dots,x_{n-1}\in\R\right\}
		\end{equation*}
		è un sottospazio vettoriale, perché ognuna delle due operazioni dà sempre come risultato un vettore con l'$n$-esima componente nulla.
	\item Dato $\R[x]$, l'insieme dei polinomi di grado non maggiore di $n$, indicato con $\R_n[x]=\{p(x)=a_0+a_1x+\dots+a_nx^n\colon a_0,a_1,\dots,a_n\in\R\}$, formano un sottospazio vettoriale di $\R[x]$.
		Infatti la somma di due poliniomi di grado massimo $n$ è ancora un polinomio di grado massimo $n$, mentre moltiplicando un polinomio per uno scalare non nullo si moltiplicano i coefficienti di ogni termine per tale scalare, quindi il grado rimane immutato.
		Moltiplicando per zero si ottiene invece un polinomio nullo, che ha ancora ovviamente grado minore di $n$.
		Lo stesso vale per $\C_n[x]\leq \C[x]$.
	\item L'insieme $\cont{}(\R)$ delle funzioni definite da $\R$ a $\R$ e continue è un sottospazio vettoriale dello spazio delle funzioni $f\colon\R\to\R$.
		Infatti sommando due funzioni continue si ottiene una funzione continua, e ovviamente anche moltiplicando una funzione continua per uno scalare.
\end{itemize}

\begin{teorema} \label{t:intersezione-sottospazi}
	Sia $V$ uno spazio vettoriale su un campo $K$ e sia $\{W_i\}_{i\in I}$ un insieme di sottospazi vettoriali di $V$.
	Allora la loro intersezione	$\bigcap_{i\in I}W_i$ è ancora un sottospazio vettoriale di $V$.
\end{teorema}
\begin{proof}
	Siano $w_1,w_2\in\bigcap_{i\in I}W_i$.
	Allora $\forall i\in I$, $w_1$ e $w_2$ appartengono a $W_i$ (appartengono a tutti i sottospazi).
	Poiché i $W_i$ sono sottospazi vettoriali, allora accade sempre che $\forall i\in I$, $w_1+w_2\in W_i$, quindi appartengono anche a $\bigcap_{i\in I}$.
	Un ragionamento analogo si effettua per il prodotto per scalare.
	Quindi $\bigcap_{i\in I}$ è un sottospazio vettoriale di $V$.
\end{proof}
Il teorema non vale se al posto dell'intersezione si effettua l'unione dei $W_i$: ad esempio le due rette $x=0$ e $y=x$, rappresentate in forma vettoriale come $\big\{\big(\begin{smallmatrix} x\\0 \end{smallmatrix}\big)\colon x\in\R\big\}$ e $\big\{\big(\begin{smallmatrix} x\\x \end{smallmatrix}\big)\colon x\in\R\big\}$, sono banalmente due sottospazi vettoriali di $\R^2$.
Prendendo però un elemento del primo e uno del secondo, $\big(\begin{smallmatrix} 1\\0 \end{smallmatrix}\big)$ e $\big(\begin{smallmatrix} 1\\1 \end{smallmatrix}\big)$, sommandoli si ottiene $\big(\begin{smallmatrix} 2\\1 \end{smallmatrix}\big)$ che non appartiene all'unione dei due sottospazi.

\begin{definizione} \label{d:sottospazio-generato}
	Siano $V$ uno spazio vettoriale su $K$ e $S\subseteq V$ un insieme non vuoto.
	Si dice \emph{sottospazio generato} di $V$, e si indica con $\gen{S}$, un sottospazio vettoriale che soddisfa le seguenti due proprietà:
	\begin{itemize}
		\item $S\subseteq\gen{S}$;
		\item se $W\leq V$ tale che $S\subseteq W$, allora $\gen{S}\leq W$.
	\end{itemize}
\end{definizione}
\begin{teorema} \label{t:unicità-span}
	Siano $V$ uno spazio vettoriale su $K$ e $S\subseteq V$ un insieme non vuoto, esiste sempre $\gen{S}$ ed è unico.
\end{teorema}
\begin{proof}
	\textit{(Unicità)} Siano $Z_1\neq Z_2$ due sottospazi vettoriali di $V$ che soddisfino la definizione \ref{d:sottospazio-generato} di sottospazio generato.
	Poiché per tale definizione $S\subseteq Z_1$, dato $W$ sottospazio di $V$ e $S\subseteq W$ segue che $Z_1\leq W$.
	Si ripete lo stesso ragionamento per $Z_2$, per cui anche $Z_2\leq W$, quindi sia $Z_1$ che $Z_2$ sono sottospazi vettoriali di $V$ come di $W$, quindi sostituendoli si conclude che $Z_1\leq Z_2$ ma anche $Z_2\leq Z_1$.
	Poiché un sottospazio vettoriale è anche un sottoinsieme, segue che $Z_1\subseteq Z_2$ e $Z_2\subseteq Z_1$, ossia $Z_1\equiv Z_2$.

	\textit{(Esistenza)} Sia $\gen{S}=\bigcap_{i=I}Z_i$ dove $\{Z_i\}_{i\in I}$ sono tutti sottospazi vettoriali di $V$ che includono $S$; ogni $Z_i$ non è vuoto perché include $S$.
	Sicuramente $\gen{S}$ è, a sua volta, un sottospazio di $V$ per il teorema \ref{t:intersezione-sottospazi}.
	$S$ è contenuto in ogni $Z_i$, quindi è incluso anche in $\gen{S}=\bigcap_{i\in I}Z_i$.
	Inoltre, sia $W$ un sottospazio di $V$ tale che $S\subseteq W$.
	Sicuramente, poiché $\gen{S}$ è un sottospazio vettoriale, una combinazione lineare di elementi di $S$ lo è anche di elementi di $\gen{S}$ quindi è un elemento di $\gen{S}$; ora, tutti gli elementi di $S$ sono anche elementi di $W$, quindi $\gen{S}$ soddisfa la definizione \ref{d:sottospazio-vettoriale} e dunque $\gen{S}\leq W$.
	Allora uno spazio che soddisfi la definizione \ref{d:sottospazio-generato} si può sempre costruire.
\end{proof}

Definiamo ora la somma di sottospazi come l'insieme $U+W=\{  u+  w\colon  u\in U,   w\in W\}$: esso è un sottospazio vettoriale, infatti
\begin{itemize}
	\item $(  u_1+  w_1)+(  u_2+  w_2)=(  u_1+  u_2)+(  w_1+  w_2)\in U+W$;
	\item $\lambda(  u+  w)=\lambda  u+\lambda  w\in U+W$.
\end{itemize}
Dimostriamo inoltre che $U+W$ è lo spazio generato dall'unione dei due sottospazi, seguendo la definizione \ref{d:sottospazio-generato}.
\begin{teorema}
	Siano $U,W$ sottospazi vettoriali di $V$ su un campo $K$. Allora $\gen{U\cup W}\equiv U+W$.
\end{teorema}
\begin{proof}
	Ogni $  u\in U$ si può scrivere come $  u+0_W=  u+  0_V$ che quindi appartiene a $U+W$, quindi $U\subseteq U+W$ e analogamente $W\subseteq U+W$, quindi $U\cup W\subseteq U+W$.
	Consideriamo un sottospazio vettoriale $T$ di $V$ che includa $U\cup W$: ogni elemento $  u+  w$ appartiene anche a $T$ per qualunque $  u$ e $  w$, ma allora $U+W$ è un sottoinsieme di $T$ oltre che uno spazio vettoriale, e ciò lo rende un sottospazio vettoriale di $T$.
	Abbiamo allora dimostrato che $U+W$ soddisfa la definizione \ref{d:sottospazio-vettoriale}, perciò $U+W=\gen{U\cup W}$.
\end{proof}

\section{Sistemi di generatori} \label{sec:sistemi-generatori}
Sia $V$ uno spazio vettoriale su $K$, e $S\subseteq V$ un insieme non vuoto.
Le combinazioni lineari (sempre finite!) di elementi di $S$ sono definite come
\begin{equation*}
	\sum_{i=1}^n\lambda_i  s_i=\lambda_1  s_1+\lambda_2  s_2+\dots+\lambda_n  s_n,
\end{equation*}
con $\lambda_i\in K$, $  s_i\in S$ e $n\in\N$.

\begin{teorema}
	Sia $V$ uno spazio vettoriale su $K$, e $S\subseteq V$ non vuoto. Allora
	\begin{equation*}
		\gen{S}=\left\{\sum_{i=1}^n\lambda_i  s_i\colon \lambda_i\in K,   s_i\in S, i=1,2,\dots,n, n\in\N\right\}.
	\end{equation*}
\end{teorema}
\begin{proof}
	Questo particolare $\gen{S}$ deve soddisfare la definizione \ref{d:sottospazio-generato}:
	\begin{itemize}
		\item i $  s_i$ appartengono a $S$, e possiamo esprimerli come $  s_i=1_K  s_i$ quindi $S\subseteq\gen{S}$;
		\item se $W\leq V$ e $S\subseteq W$, allora dato che $\gen{S}\supseteq S$ se prendiamo una combinazione lineare di due elementi di $S$, lo è anche di elementi di $W$, e poiché il risultato è sempre un elemento di $\gen{S}$ quest'ultimo è un sottospazio vettoriale di $W$.
	\end{itemize}
\end{proof}

\begin{definizione}
	Sia $V$ uno spazio vettoriale sul campo $K$ e sia $S\subseteq V$ un insieme non vuoto. $S$ è detto \emph{sistema di generatori} per $V$ se $\gen{S}=V$.
\end{definizione}
Con questa definizione possiamo studiare anziché l'intero spazio vettoriale $V$ solo un suo sottoinsieme.
\paragraph{Esempi}
\begin{itemize}
	\item Come già detto, i vettori di $\R^n$ sono definiti dalle loro coordinate, quindi possono essere scritti come combinazioni lineari di questi elementi: allora
		\begin{equation*}
			\R^n=\gen{
				\begin{pmatrix}
					1\\0\\\vdots\\0
				\end{pmatrix},
				\begin{pmatrix}
					0\\1\\\vdots\\0
				\end{pmatrix},\dots,
				\begin{pmatrix}
					0\\0\\\vdots\\1
				\end{pmatrix}
			}
		\end{equation*}
		I vettori dello spazio generatore sono a tutti gli effetti dei versori di $\R^n$, in questo esempio sono i versori allineati con gli assi cartesiani.
	\item $\R[x]$ è generato da $\{1,x,x^2,\dots,x^n,\dots\}$; questo insieme è infinito, perché non esiste un polinomio ``di grado massimo''.
		Ogni $  x\in\R[x]$ è determinato da una combinazione lineare di questi componenti, in modo univoco.
	\item In $\R^2$ si può individuare il sistema di generatori $\gen{\big(\begin{smallmatrix} 0\\1 \end{smallmatrix}\big),\big(\begin{smallmatrix} 1\\0 \end{smallmatrix}\big),\big(\begin{smallmatrix} 1\\1 \end{smallmatrix}\big)}$.
		Con questo insieme però si può scrivere l'elemento $\big(\begin{smallmatrix} 2\\2 \end{smallmatrix}\big)$ in due modi diversi, ossia come $2\big(\begin{smallmatrix} 0\\1 \end{smallmatrix}\big)+2\big(\begin{smallmatrix} 1\\0 \end{smallmatrix}\big)+0\big(\begin{smallmatrix} 1\\1 \end{smallmatrix}\big)$ ma anche come $0\big(\begin{smallmatrix} 0\\1 \end{smallmatrix}\big)+0\big(\begin{smallmatrix} 1\\0 \end{smallmatrix}\big)+2\big(\begin{smallmatrix} 1\\1 \end{smallmatrix}\big)$.
		Questo sistema di generatori quindi non permette di scrivere in maniera univoca i vettori di $\R^2$.
\end{itemize}

\begin{definizione} \label{d:dipendenza-lineare}
	Sia $V$ uno spazio vettoriale su $K$, e $\{  v_i\}_{i\in I}\subseteq V$.
	Si dice che l'insieme $\{v_i\}$ è \emph{linearmente dipendente} se esiste $I_0\subseteq I$, di cardinalità $n$ finita, e un insieme di scalari $\{\lambda_1,\lambda_2,\dots,\lambda_n\}\in K\setminus\{0_K\}$ tali per cui
	\begin{equation*}
		\sum_{i=1}^n\lambda_i  v_i=0_V,
	\end{equation*}
	dove $\{  v_i\}_{i=1}^n$ è una numerazione di $\{  v_i\}_{i\in I}$.
\end{definizione}
In parole povere, un insieme è linearmente dipendente se esiste almeno una combinazione lineare (con i coefficienti non tutti nulli) dei suoi componenti che dia lo zero dello spazio.
Per l'ultimo degli esempi precedenti l'insieme $\big\{\big(\begin{smallmatrix} 0\\1 \end{smallmatrix}\big),\big(\begin{smallmatrix} 1\\0 \end{smallmatrix}\big),\big(\begin{smallmatrix} 1\\1 \end{smallmatrix}\big)\big\}$ è linearmente dipendente.

Ovviamente, un sistema che non è linearmente dipendente si dice \emph{linearmente indipendente}, o anche \emph{libero}.

\begin{definizione} \label{d:indipendenza-lineare}
	Un insieme finito di vettori $\{  v_1,  v_2,\dots,  v_k\}$ si dice \emph{linearmente indipendente}, se in ogni combinazione lineare dei $k$ vettori che produce $0_V$ i coefficienti sono tutti nulli:
	\begin{equation*}
		\lambda_1  v_1+\lambda_2  v_2+\dots+\lambda_k  v_k=0_V\ \then\ \lambda_1=\lambda_2=\dots=\lambda_k=0_K.
	\end{equation*}
	Un insieme infinito di vettori $\{  v_i\}_{i\in I}$ (con $I$ quindi anche di cardinalità infinita) è linearmente indipendente se $\forall J\subseteq I$ di cardinalità finita $\{  v_j\}_{j\in J}$ è linearmente indipendente (cioè se lo è ogni suo sottoinsieme).
\end{definizione}
Alcuni esempi di insiemi linearmente indipendenti:
\begin{itemize}
	\item il sistema che genera $K_n[x]$, ossia $\{1,x,x^2,\dots,x^n\}$, è linearmente indipendente perché un polinomio è identicamente nullo se e solo se tutti i coefficienti dei vari termini sono nulli.
		Lo stesso vale per i polinomi di grado non limitato di $K[x]$, poiché la definizione è verificata da ``blocchi'' di termini.
	\item l'insieme $\{  v,  w,0_V,  z\}\subset V$ spazio vettoriale su $K$ non lo è, poiché $0_K  v+0_K  w+1_K0_V+0_K  z=0_V$ anche se uno dei coefficienti, $1_K$, non è nullo.
\end{itemize}
Il seguente teorema indica un modo più semplice di verificare questa definizione.
\begin{teorema}
	Un insieme di vettori $\{  v_i\}_{i\in i}\subset V$ è linearmente dipendente se e solo se almeno uno di essi è una combinazione lineare di un numero finito dei rimanenti.
\end{teorema}
\begin{proof}
	Sia dato l'insieme, linearmente dipendente, $\{  v_i\}_{0<i\leq k}$: esiste una combinazione lineare $\sum_{n=1}^k\lambda_n  v_n$ nulla senza che tutti i $\lambda_n$ siano nulli.
	Trascurando nella serie gli eventuali termini nulli, rimangono un numero finito di termini tali che ad esempio $\lambda_1  v_1=-\lambda_2  v_2-\dots-\lambda_n  v_n$.
	Poiché $\lambda_1$ non è nullo, esiste il suo inverso rispetto al rapporto, $(\lambda_1)^{-1}$, e moltiplicando la precedente equazione per questo risulta che il primo termine è $(\lambda_1)^{-1}(\lambda_1  v_1)=(\lambda_1^{-1}\lambda_1)  v_1=1_K  v_1=v_1$, allora
	\begin{equation*}
	  v_1=(\lambda_1^{-1})(-\lambda_2  v_2-\dots-\lambda_n  v_n),
	\end{equation*}
	che è quindi combinazione lineare degli altri vettori dell'insieme.

	Sia $  v^*\neq 0_V$ un vettore dell'insieme dato, combinazione lineare (in cui quindi i coefficienti non possono essere tutti nulli) di alcuni dei vettori rimanenti, quindi
	\begin{equation*}
		  v^*=\mu_1  v_1+\mu_2  v_2+\dots+\mu_r  v_r.
	\end{equation*}
	Portando tutto al primo termine risulta $  v^*-\mu_1  v_1-\mu_2  v_2-\dots-\mu_r  v_r=0_V$ sebbene non siano tutti nulli.
\end{proof}

\section{Basi e dimensioni} \label{sec:basi-dimensioni}
\begin{definizione} \label{d:base}
	Si chiama \emph{base} di uno spazio vettoriale $V$ ogni sistema $S$ linearmente indipendente che genera $S$.
\end{definizione}
	Un sistema di generatori esiste sempre per ogni spazio non vuoto: semmai si può prendere lo spazio stesso; non è sempre certa, però, l'esistenza di un insieme linearmente indipendente.
\begin{teorema}
	Sia $\{  e_i\}_{i\in I}$ un insieme di $V$.
	Esso è una base di $V$ se e solo se ogni elemento $  v\in V$ (non nullo) si può scrivere in modo univoco come combinazione lineare finita, a coefficienti non nulli, di elementi di $\{  e_i\}_{i\in I}$.
\end{teorema}
Gli elementi $  v$ non devono essere nulli, perché $0_V$ si può scrivere come combinazione lineare di \emph{qualunque} sistema di vettori; inoltre i coefficienti della combinazione non devono essere nulli, altrimenti si potrebbe affermare che $  v=a  e_1+b  e_2$ ma anche $  v=a  e_1+b  e_2+0_K  e_3+\dots+0_K  e_n+\dots$ quanto si vuole.
\begin{proof}
	Dimostriamo che la condizione è necessaria.
	Sia $\{  e_i\}_{i\in I}$ una base di $V$: allora genera tutto $V$.
	Preso un elemento $  v\in V$ non nullo, si può scrivere come la combinazione lineare finita
	\begin{equation} \label{eq:dim-base1}
		  v=\sum_{i\in I_0}\lambda_i  e_i,
	\end{equation}
	con $I_0\subset I$ di cardinalità finita.
	Dimostriamo che questa scrittura è unica: supponiamo che $\forall i\in I_0$ $\lambda_i\neq 0_K$ (eventuali termini nulli nella combinazione lineare si trascurano); allora esiste anche 
	\begin{equation} \label{eq:dim-base2}
		  v=\sum_{j\in J_0}\mu_j  e_j
	\end{equation}
	con $\mu_j\neq 0_K$ $\forall j\in J_0\subset I$ e di cardinalità finita.
	Sommando gli opposti della \eqref{eq:dim-base2} alla \eqref{eq:dim-base1} si ha
	\begin{equation*}
		\sum_{i\in I_0}\lambda_i  e_i+\sum_{j\in J_0}-\mu_j  e_j=0_V.
	\end{equation*}
	Sia $I_0\subseteq J_0$, e ipotizziamo che esista $j_0$ appartenente a $J_0$ ma non a $I_0$.
	Allora l'elemento $  e_{j_0}$, nella combinazione, è associato al coefficiente $-\mu_{j_0}\neq 0_K$ (quindi esiste il suo reciproco).
	Portando $-\mu_{j_0}  e_{j_0}$ al secondo membro dell'uguaglianza e moltiplicando per il reciproco di $\mu_{j_0}$ risulta
	\begin{equation*}
		\sum_{i\in I_0}(\lambda_i\mu_{j_0}^{-1})  e_i+\sum_{j\in J_0}(\mu_j\mu_{j_0}^{-1})  e_j=  e_{j_0},
	\end{equation*}
	cioè uno degli elementi di $\{  e_i\}$ è espresso come combinazione lineare degli altri, vale a dire che la base è linearmente dipendente, il che è assurdo perché è una base: quindi non può esistere un $j_0$ che appartiene a $J_0$ ma non a $I_0$.
	Ponendo $J_0\subseteq I_0$ si ottiene allo stesso modo un altra contraddizione.
	Allora non può che essere $I_0\equiv J_0$, ma ciò significa che
	\begin{equation*}
		\sum_{i\in I_0}(\lambda_i-\mu_i)  e_i=0_V,
	\end{equation*}
	cui segue che $\lambda_i=\mu_i$ $\forall i\in I_0$, cioè le \eqref{eq:dim-base1} e \eqref{eq:dim-base2} sono identiche e dunque la scrittura di $  v$ in termini della base è unica.

	Mostriamo ora che la condizione è anche sufficiente: innanzitutto, $\gen{\{  e_i\}_{i\in I}}=V$ perché per ipotesi possiamo scrivere ogni vettore di $V$ come combinazione lineare di elementi di questo insieme.
	Supponiamo che esista $I_0\subset I$ di cardinalità finita, per cui
	\begin{equation} \label{eq:dim-base3}
		\sum_{i\in I_0}\lambda_i  e_i=0_V,
	\end{equation}
e come prima che $\lambda_i\neq 0_K$ $\forall i\in I_0$.
	Considerando un $i^*\in I_0$, $\lambda_{i^*}$ non è nullo, quindi moltiplicando la \eqref{eq:dim-base3} per il suo inverso si trova
	\begin{equation*}
		\sum_{i\in I_0}(\lambda_{i^*}^{-1}\lambda_i)  e_i=0_V.
	\end{equation*}
	Isolando il termine in $i^*$ si ottiene poi che
	\begin{equation*}
		\sum_{i^*\neq i\in I_0} (\lambda_{i^*}^{-1}\lambda_i)  e_i=-  e_{i^*}
	\end{equation*}
	vale a dire che $  e_{i^*}$ si esprime come combinazione lineare di altri elementi dell'insieme.
	Questo elemento però non è unico, dato che il ragionamento vale per qualsiasi $i\in I_0$ preso volta per volta.
	Allora è assurdo che esista un tale $I_0$, cioè che per tali $i\in I_0$ la combinazione lineare sia nulla pur non avendo tutti i coefficienti nulli; dunque l'insieme è linearmente indipendente, e dato che genera $V$ è una sua base.
\end{proof}
In $\R^3$ ad esempio ogni punto è univocamente individuato da un vettore $  v$ che è esprimibile nelle sue (tre) coordinate come $  v=xe_1+ye_2+ze_3$, infatti l'insieme $\{e_1,e_2,e_3\}=\{(1,0,0),(0,1,0),(0,0,1)\}$ è una base di $\R^3$ (detta comunemente \emph{base canonica}).
I coefficienti di ogni termine formano le componenti del vettore.

\begin{definizione}
	Sia $V$ uno spazio vettoriale su $K$: esso si dice di dimensione finita se ammette un sistema finito di generatori, altrimenti si dice di dimensione infinita.
\end{definizione}
\begin{teorema} \label{t:esistenza-base}
	Sia $G\subset V$ un sistema di generatori di $V$ finito.
	Se $S$ è un sottoinsieme di $G$ linearmente indipendente, allora esiste una base $B$ di $V$ tale da comprendere il sistema $S$ e che sia inclusa in $G$ (cioè $S\subseteq B\subseteq G$).
\end{teorema}
\begin{proof}
	Si supponga che $V$ abbia dimensione finita: allora esiste un sistema di generatori $G$.
	Escludiamo il caso in cui $V\equiv\{0_V\}$ perché non esisterebbe nemmeno una base.
	Esiste $  e\in G$, non nullo, che ovviamente forma un sistema linearmente indipendente poiché $\lambda  e\neq 0_V$ $\forall\lambda\neq 0_K$.
	Inoltre, per come si è scelto $  e$, $S\equiv\{  e\}\subseteq G$.
	Quindi esiste sempre una base di $V$ per cui $\{  e\}\subseteq B\subseteq G$.

	Indichiamo con $S_n$ il fatto che nell'insieme linearmente indipendente $S$ ci siano $n$ vettori.
	Potrebbe essere che $\gen{S_n}\equiv V$, ma allora possiamo scegliere subito $S_n$ come base per $V$, e poiché $B=S_n\subseteq G$ il teorema è dimostrato.
	Sia allora $\gen{S_n}\lneqq V$: deve esistere $  x_{n+1}\in G$, ma $\notin\gen{S_n}$, perché altrimenti dato che $\gen{S}\supseteq G$ si avrebbe che $\gen{S_n}\equiv V$.
	Definiamo $S_{n+1}=S_n\cup\{  x_{n+1}\}$ (quindi l'insieme ha $n+1$ elementi), per cui sicuramente vale $S_n\subseteq S_{n+1}\subseteq G$. Questo $S_{n+1}$ è un insieme linearmente indipendente, altrimenti avremmo che $  x_{n+1}\in\gen{S_n}$.
	Se $\gen{S_{n+1}}=V$ il teorema è dimostrato, altrimenti procediamo aggiungendo un altro elemento di $G\setminus\gen{S_{n+1}}$.
	Iteriamo il processo per $n+2$, $n+3$ e così via, fino a quando si esauriscono gli elementi di $G$, ottenendo la relazione
	\begin{equation*}
		S_n\subseteq S_{n+1}\subseteq S_{n+2}\subseteq\dots\subseteq G.
	\end{equation*}
	La dimensione di $G$ è finita, quindi prima o poi gli elementi da aggiungere termineranno: esisterà $k\in\N$ per cui $\gen{S_{n+k}}=\gen{G}$, e anche in questo caso abbiamo trovato che $S_{n+k}$ è base di $V$.
\end{proof}
Sempre in $\R^3$, per esempio, una delle possibili basi è quella composta dai tre versori $\{\ii,\jj,\kk\}=\{(1,0,0),(0,1,0),(0,0,1)\}$, ma anche $\{  e_1,  e_2,  e_3\}=\{\ii,\jj,(\jj+\kk)\}$ è un'altra base.
In effetti ruotando $\ii$, $\jj$ e $\kk$ di un angolo qualsiasi si ottiene un altra base, e se ne ottengono ancora delle altre moltiplicando per degli scalari (anche differenti) i tre versori.
Quindi le basi di uno spazio vettoriali sono infinite; quello che non cambia è il numero di elementi di queste basi, che è sempre costante (in questo esempio, la base è sempre composta da tre vettori).
\begin{corollario}
	Ogni spazio vettoriale $V\neq\{0_V\}$ di dimensione finita ammette almeno una base.
\end{corollario}
\begin{teorema} \label{t:base-dimensione}
	Sia $V$ uno spazio vettoriale di dimensione finita, contenente una base di $n$ vettori.
	Allora:
	\begin{enumerate}
		\item ogni sistema linearmente indipendente $S$ di $n$ vettori è una base di $V$;
		\item ogni sistema $U$ di $m>n$ vettori è linearmente dipendente;
		\item ogni sistema $W$ di $m<n$ vettori non può generare $V$, cioè $\gen{W}\neq V$;
		\item ogni sistema $T$ di $n$ vettori per cui $\gen{T}=V$ è una base.
	\end{enumerate}
\end{teorema}
\begin{proof}
\begin{verbatim}

\end{verbatim}
	\begin{enumerate}
		\item Siano $\{  e_i\}_{i=1}^n$ una base di $V$, e $S=\{  f_i\}_{i=1}^n$ un insieme finito linearmente indipendente.
			Allora
			\begin{equation*}
				  f_1=\sum_{i=1}^n\lambda_i  e_i.
			\end{equation*}
			Poiché $S$ è linearmente indipendente, almeno un $\lambda_i$ non è nullo, quindi $  f_1\neq 0_V$, e riordinando i vettori nella combinazione possiamo supporre che sia $\lambda_1\neq 0_K$: in questo modo $\lambda_1  e_1\neq 0_V$.
			Portando quest'ultimo termine al secondo membro e moltiplicando per $\mu_1=\lambda_1^{-1}$ risulta con opportuni $\mu_i\in K$ che
			\begin{equation*}
				  e_1=\mu_1  f_1+\sum_{i=2}^n\mu_i  e_i.
			\end{equation*}
			Ogni vettore di $V$ è una combinazione lineare di elementi di $\{  e_i\}_{i=1}^n$, ma sostituendo $  e_1$ con l'espressione trovata sopra abbiamo $\forall  v\in V$
			\begin{equation*}
				  v=\sum_{i=1}^n\lambda_i  e_i=\lambda_1\bigg(\mu_1  f_1+\sum_{i=2}^n\mu_i  e_i\bigg)+\sum_{i=2}^n\lambda_i  e_i
			\end{equation*}
			che quindi può essere espresso anche come combinazione lineare di $\{  f_1,  e_2,\dots,  e_n\}$ anziché degli $\{  e_i\}_{i=1}^n$, quindi anche l'insieme $\{  f_1,  e_2,\dots,  e_n\}$ è un sistema di generatori di $V$.
			Dunque troviamo anche che
			\begin{equation*}
				  f_2=\sigma_1  f_1+\sum_{i=1}^n\sigma_i  e_i.
			\end{equation*}
			Almeno uno dei $\sigma_i$ non è nullo, altrimenti sarebbe che $  f_2=\sigma_1  f_1$ che contraddice l'indipendenza lineare degli $  f_i$.
			Supponendo $\sigma_2\neq 0_K$, si esplicita $  e_2$ moltiplicando per $\sigma_2^{-1}$, ottenendo
			\begin{equation*}
				  e_2=\rho_1  f_1+\rho_2  f_2+\sum_{i=3}^n\rho_i  e_i.
			\end{equation*}
			L'insieme $\{  f_1,  f_2,  e_3,\dots,  e_n\}$ è ancora un sistema di generatori di $V$.
			Si itera il procedimento ottenendo alla fine che $\{  f_1,  f_2,\dots,  f_n\}$ è ancora un sistema di generatori per $V$, e dunque ne è una base dato che è linearmente indipendente. 
		\item Sia $U$ con $m>n$ elementi linearmente indipendente, e si prenda $U'\subset U$ tale che abbia $n$ elementi (dunque $U\setminus U'\neq\emptyset$).
			Per il punto 1 $U'$ è una base di $V$, quindi i vettori di $U'$ generano anche quelli di $U\setminus U'$.
			Ciò contraddice l'indipendenza lineare di $U$, che deve essere quindi linearmente dipendente.
		\item Sia $W$ con $m<n$ elementi un sistema di generatori di $V$: allora deve esistere una base con al più $m$ vettori, tanti quanti ce ne sono in $W$.
			Per il punto precedente, la base $\{  e_i\}_{i\in I}$ (considerata nel punto 1) ha più vettori della base estratta da $W$, quindi sarebbe linearmente dipendente, che è assurdo.
			Allora $W$ non può essere un sistema di generatori di $V$.
		\item Se $\gen{T}=V$, $T$ deve avere almeno $n$ elementi per il punto 3; allora esiste $T'\subseteq T$ che è una base di $V$.
			Se $T'$ avesse meno di $n$ elementi, contraddirrebbe il punto 3 prima citato, quindi deve averne esattamente $n$, perciò $T\equiv T'$, e $T$ è linearmente indipendente.
			Poiché genera $V$, $T$ ne è anche una base.
	\end{enumerate}
\end{proof}
\begin{corollario}
	Sia $V$ uno spazio vettoriale di dimensione finita, contenente una base di $n$ vettori.
	Ogni altra base $V$ ha a sua volta esattamente $n$ vettori.
\end{corollario}
\begin{definizione} \label{d:dimensione}
	Dato uno spazio vettoriale $V\neq\{0_V\}$ su $K$ di dimensione finita, si dice \emph{dimensione} di $V$ su $K$ il numero di vettori di una sua base qualunque.
\end{definizione}
La dimensione di $V$ (su $K$) si indica con $\dim_KV$ o anche solo, se non ci sono ambiguità, con $\dim V$. Convenzionalmente, allo spazio contenente soltanto $\{0_V\}$ si assegna la dimensione 0.
\begin{teorema}
	Sia $V$ uno spazio vettoriale, e $W$ un suo sottospazio.
	Se $\dim V$ è finita, allora $\dim W\leq\dim V$.
\end{teorema}
\begin{proof}
	Nel caso banale in cui $W=\{0_V\}$, la sua dimensione è 0 quindi è ovviamente minore o uguale della dimensione di $V$, qualunque essa sia.
	Se invece $W$ non contiene soltanto il vettore nullo, una base qualunque di $W$ è un sistema linearmente indipendente anche in $V$.
	Siccome $W\neq\{0_V\}$, esiste un vettore $  w_1\in W$ non nullo.
	Se $\gen{  w_1}=W$ allora $\{  w_1\}$ è una base di $W$ da cui si ottiene subito la tesi: infatti $  w_1$ appartiene anche a $V$ che dunque ha almeno dimensione 1.
	Altrimenti $\gen{  w_1}\neq W$ ma comunque esiste un altro vettore di $W$, $  w_2\notin\gen{  w_1}$ (quindi appartenente a $W\setminus\gen{  w_1}$).
	Sia $S=\{  w_1,  w_2\}$: esso è un sistema linearmente indipendente per come abbiamo scelto $  w_2$.
	%Nella combinazione lineare $\lambda_1  w_1+\lambda_2  w_2=0_V$ non può essere $\lambda_1=0$ perché $\{  w_1\}$ è un sistema libero, mentre ugualmente $\lambda_2\neq 0$ perché\dots?
	Se $\gen{S}=W$, come nel caso precedente abbiamo dimostrato il teorema: per teorema precedente (\ref{t:base-dimensione}) $V$ ha due vettori linearmente indipendenti dunque deve essere $\dim V\geq 2=\dim W$.
	Se invece $\gen{S}\neq W$, si consideri un altro $  w_3\in W\setminus\gen{  w_1,  w_2}$ e si ripete il ragionamento compiuto in precedenza.
	Ogni volta si trova un nuovo insieme $S=\{  w_1,  w_2,\dots,  w_n\}$, ancora linearmente indipendente in $V$.
	Per il punto 3, sempre del teorema \ref{t:base-dimensione}, $\dim V\geq\abs{S}$, quindi $W$ ha dimensione finita, inoltre $n=\dim W=\abs{S}\leq\dim V$.
\end{proof}

\begin{teorema}\label{estensione-base}
	Sia $V$ uno spazio vettoriale di dimensione finita, $W$ un suo sottospazio e $B_W$ una base di $W$.
	Allora tale base si può estendere per formare una base di $V$, cioè $\exists B_V\colon B_W\subseteq B_V$.
\end{teorema}
\begin{proof}
	$B_W$ è linearmente indipendente in $W$ in quanto è una base, e lo è quindi anche in $V$ (quindi $\dim V\geq\abs{B_W}$), che deve avere un sistema finito $G$ di generatori.
	Allora $B_W\cup G$ è ancora un sistema di generatori per $V$, ed è anche finito.
	Da questo insieme, che per generare $V$ deve essere tale che $\dim V\leq\abs{B_W\cup G}$, possiamo estrarre $\dim V$ elementi linearmente indipendenti.
	Poiché $B_W\subseteq B_W\cup G$ è già linearmente indipendente, esiste una base $B_V$ di $V$ contenente $B_W$ (e contenuta in $B_W\cup G$).
\end{proof}
Ad esempio, $\R$ è un sottospazio di $\R^3$: prendendo una base $\{  v\}$ del primo, si ottiene una base del secondo semplicemente aggiungendo due vettori (distinti) perpendicolari a $  v$.

\begin{definizione} \label{d:insieme-linearmente-indipentente}
	Un insieme $\{V_i\}_{i\in I}$ di sottospazi vettoriali di $V$ si dice linearmente indipendente se comunque si scelga un vettore $x_i\neq 0$ per ciascun $V_i$, l'insieme $\{x_i\}_{i\in I}$ è linearmente indipendente.
\end{definizione}
\begin{teorema}
	Un insieme $\{V_i\}_{i\in I}$ di sottospazi vettoriali di $V$ è linearmente indipendente se e solo se $\forall k\in I$ si ha che l'intersezione tra $V_k$ e lo spazio generato dai restanti $V_i$ contiene soltanto lo zero, cioè\footnote{Ricordiamo che la somma di più spazi vettoriali è lo spazio generato dalla loro unione, ossia $\sum_i V_i=\gen{\bigcup_i V_i}$.}
	\begin{equation*}
		V_k\cap\sum_{j\in I\setminus\{k\}}{V_j}=\{0\}.
	\end{equation*}
\end{teorema}
\begin{proof}
	Se i sottospazi $V_i$ sono linearmente indipendenti, e per assurdo $V_k\cap\sum_{j\in I\setminus\{k\}}{V_j}\neq\{0\}$ per qualche $k\in I$, allora esisterebbe un elemento $v_k\in V_k$ non nullo che è combinazione lineare di elementi dei restanti sottospazi, cioè $v_k=x_{i_1}+\dots+x_{i_r}$ con $\{i_1,\dots,i_r\}\subseteq I\setminus\{k\}$.
	Ma allora l'insieme $\{V_i\}_{i\in I}$ dei sottospazi non sarebbe linearmente indipendente, contraddicendo l'ipotesi, quindi l'uguaglianza deve essere vera.

	Viceversa, se prendessimo una combinazione lineare nulla di elementi uno da ciascun sottospazio
	\begin{equation}
		a_{i_1}v_{i_1}+\dots+a_{i_k}v_{i_k}=0
		\label{eqdim:sottospazi-linearmente-indipendenti}
	\end{equation}
	con $v_{i_j}\in V_{i_j}$ appartenenti tutti a sottospazi distinti, e ci fosse $a_{i^*}\ne 0$, allora potremmo scrivere
	\begin{equation}
		v_{i^*}=\sum_{j\in I\setminus\{i^*\}}c_jv_j
	\end{equation}
	ma allora $v_{i^*}\in V_{i^*}$ e anche $v_{i^*}\in\sum_{j\in I\setminus\{i^*\}}V_j$, ossia esiste un $i^*\in I$ tale per cui
	\begin{equation*}
		V_{i^*}\cap\sum_{j\in I\setminus\{i^*\}}V_j\ne\{0\}
	\end{equation*}
	che contraddice l'ipotesi.
	Dunque nella \eqref{eqdim:sottospazi-linearmente-indipendenti} si ha $a_j=0$ per ogni $j\in I$, cioè i sottospazi sono linearmente indipendenti.
\end{proof}

\begin{teorema} \label{t:unione-sottospazi-linearmente-indipendente}
	Sia $\{V_i\}_{i\in I}$ un insieme linearmente indipendente di sottospazi vettoriali di $V$.
	Se $\forall i\in I$ il sistema di vettori $S_i\subset V_i$ è linearmente indipendente, allora $\bigcup_{i\in I}S_i$ è linearmente indipendente in $V$.
\end{teorema}
\begin{proof}
	Consideriamo un sistema linearmente indipendente di vettori $S_i\subset V_i$ e ipotizziamo per assurdo che l'unione non sia linearmente indipendente.
	Questo implica che, considerando gli elementi $y^k_i\in S_i$ con $i,k\in I_0$ in cui $I_0\in I$ ha cardinalità finita, 
	\begin{equation*}
		\sum_{1\leq k \leq n_i} \lambda^k_i y_k^i = 0,
	\end{equation*}
	per ogni $\lambda^k_i$ meno un $\lambda^k_{\tilde{i}}$, corrispondente a un $\tilde{k}$.
	Posto $  x_i = \sum\limits_{1\leq k \leq n_i} \lambda^k_i x_k^i$ sappiamo che $\sum_{i\in I_0}  x_i = 0$ e che un elemento di questa sommatoria non è nullo, ma per ipotesi $  x_i\subset S_i$ che è una combinazione di vettori linearmente indipendente, allora deve essere $  x_{\tilde{i}} = 0$.
	L'unione $S$ dei vari $S_i$ è allora linearmente indipendente. 
\end{proof}

\begin{definizione} \label{d:somma-diretta}
	Uno spazio vettoriale $V$ si dice \emph{somma diretta} di un insieme di sottospazi vettoriali $\{V_i\}_{i\in I}$ se $\{V_i\}_{i\in I}$ è linearmente indipendente e se $\sum_{i\in I}V_i\equiv V$.
\end{definizione}
Per indicare che $V$ è composto dalla somma diretta degli spazi $V_i$ si usa la scrittura $V=\bigoplus\limits_{i\in I}V_i$.
\paragraph{Esempi}
\begin{itemize}
	\item Lo spazio $\R[x]$ è generato dall'insieme $\{1,x,x^2,\dots\}$.
		Posto $V_j=\gen{x^j}$, con $j\in\N_0$, si ha che
		\begin{equation*}
			\R[x]=\bigoplus_{j\in\N_0}V^j=\bigoplus_{j\in\N_0}\gen{x^j}.
		\end{equation*}
	\item In $\R^3$, siano $V_1=\gen{\begin{psmallmatrix}1\\0\\0\end{psmallmatrix},\begin{psmallmatrix}0\\1\\0\end{psmallmatrix}}$ e $V_2=\gen{\begin{psmallmatrix}0\\1\\0\end{psmallmatrix},\begin{psmallmatrix}0\\0\\1\end{psmallmatrix}}$.
		Certamente $V_1+V_2=\R^3$, ma la somma non è diretta poiché la loro intersezione è l'asse $y$.
\end{itemize}

\section{Spazi quoziente} \label{sec:spazi_quoziente}
Sia $W$ un sottospazio vettoriale di $V$.
Definiamo la relazione $  x\sim   y$, con $x,y\in V$, se $  x-  y\in W$: essa è di equivalenza, perché soddisfa le tre proprietà della definizione \ref{d:relazione-equivalenza}.
Infatti $  x-  x=0_V$ che certamente appartiene a $W$; se $  x-  y=  w\in W$ esiste, allora deve esistere il suo opposto in $W$ (ed esiste, dato che $W$ è uno spazio vettoriale) che è $-  w=  y-  x\in W$; infine se $  x-  y=  w_1$ e $  y-  z=  w_2$ sono vettori di $W$, allora sommandoli si ottiene $  w_1-  w_2=  x-  y+  y-  z=  x-  z$ che poiché $W$ è uno spazio vettoriale, vi appartiene.
Prendiamo un elemento $  a\in V$: la sua classe di equivalenza è formata da tutti quegli elementi di $V$ tali che $  a-  v\in W$, cioè
\begin{equation*}
	[  a]=\{  x\in V\colon  x-  a=  w\in W\}=\{  w+  a\colon  w\in W\}.
\end{equation*}
Questa classe si indica anche come $W+  a$, che si può leggere come l'insieme degli elementi che sono somma di un elemento di $W$ e di $  a$ (e che danno un elemento di $V$): essa si chiama anche \emph{laterale} destro\footnote{In questo caso il laterale è detto \emph{destro} perché il rappresentante $  a$ si trova alla destra dell'operazione. Ovviamente esistono anche laterali sinistri, che in questo caso sarebbero della forma $  a+W$. Trovandoci in spazi vettoriali, però, la somma è commutativa quindi non ha senso distinguere i due casi.} di $W$ in $V$, con rappresentante $  a$.
L'insieme di queste classi di equivalenza distinte, come appena descritte, di $W$ in $V$ si indica con $V\quot W$ e si chiama \emph{spazio quoziente}.

Nello spazio quoziente possiamo definire la somma, che indicheremo con il simbolo $\oplus$, di due laterali come
\begin{equation*}
	(W+  a)\oplus(W+  b)=W+(  a+  b).
\end{equation*}
Se anche fosse $[  a]\equiv[  a']$ e $[  b]\equiv[  b']$, non è detto a priori che si abbia $  a+  b=  a'+  b'$ come conseguenza della proprietà transitiva\footnote{Ovviamente con $  a\neq  a'$ e $  b\neq  b'$, altrimenti sarebbe ovvia: è sufficiente a questo scopo che $  a\sim  a'$, e lo stesso per $  b$.}.
Perché ciò accada serve un'ulteriore condizione, che la relazione sia una \emph{relazione di congruenza}, ossia che sia compatibile con le operazioni in $V$.
Se da $  x\sim  x'$ e $  y\sim  y'$ seguono le relazioni $  x+  y\sim   x'+  y'$ e $\lambda  x\sim\lambda  x'$, allora la relazione $\sim$ è una relazione di congruenza nello spazio vettoriale: mostriamo che la relazione da noi usata per definire lo spazio quoziente lo è.
% Magari aggiungere un enunciato!
\begin{proof}
	Se $  x\sim  x'$, allora $  x-  x'=  w_1\in W$, e analogamente $  y-  y'=  w_2\in W$.
	Sommandoli, si ottiene $  w_1+  w_2=  x-  x'+  y-  y'=(  x+  y)-(  x'+  y')$ che essendo $W$ un sottospazio vettoriale vi appartengono: allora $  x+  y\sim  x'+  y'$.
	Anche per il prodotto con uno scalare $\lambda$, si ottiene analogamente che se $  x-  x'=  w\in W$, allora sempre poiché $W$ è un sottospazio vettoriale si ha che $\lambda  x-\lambda  x'=\lambda  w$ che appartiene a $W$.
\end{proof}
Come per $\oplus$, definiamo anche l'operazione analoga al prodotto scalare, che indichiamo con $\odot$:
\begin{equation*}
	\lambda\odot(W+  a)=W+\lambda  a.
\end{equation*}

La terna $(V\quot W,\oplus,\odot)$ è uno spazio vettoriale su $K$, dove $V$ è a sua volta uno spazio vettoriale sul medesimo campo.
Infatti, oltre alle proprietà appena dimostrate, esiste l'elemento neutro rispetto a $\oplus$, che è $W+0_V$ (si indica anche solamente con $W$), e l'opposto, che è $-(W+  a)=W+(-  a)$.

\section{Algebre}
\begin{definizione} \label{d:algebra}
	Si definisce \emph{algebra} sul campo $K$ uno spazio vettoriale $A$ sul campo $K$ munito di un'ulteriore operazione interna di prodotto, che sia associativo e distributivo rispetto alla somma, ossia tale per cui
	\begin{itemize}
		\item $\forall v,w,z\in A$ si ha: $(v w) z = v (w z)$,
		\item $\forall v,w,z\in A$ e $\forall \lambda\in K$ si ha $(\lambda v + w)z = \lambda v z + w z$.
	\end{itemize}
\end{definizione}
Un'algebra A si dice  \emph{commutativa} se il prodotto è commutativo, si dice \emph{con unità} se $\exists I_a\colon \forall a\in A \text{ si ha } a I_a = I_a a = a$.

\end{document}
